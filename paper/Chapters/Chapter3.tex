% Chapter Template

\chapter{Results} % Main chapter title

\label{Chapter 3} % Change X to a consecutive number; for referencing this chapter elsewhere, use \ref{ChapterX}




%----------------------------------------------------------------------------------------
%	SECTION 1
%----------------------------------------------------------------------------------------

\subsection{Model selection, and interpretation}\vspace{-0.4cm}

A series of traditional LCA of the responses to carbohydrate intake within 7 time slots of day was first examined. These initial analyses ignored the clustering of observation days within participants of the survey. \textbf{Table \ref{tab:mixmodels}} shows the latent class solutions for one to five classes (see rows under the Fixed effects model section). The BIC declines with the number of day level classes increases. However, the improvement of BIC dropped to less than 1000 from 3 classes to 4 classes solutions (658.9) and from 4 classes to 5 classses solutions (361.7). Entropy index indicates that the 4 classes model could explain about 51\% percent of the data, while \textit{p} values of Lo-Mendell-Rubun LRT suggest that the more classes we fit, the better model we will have until up to 6 classes (\textit{p} = 0.06 and is not shown in the table). From the parsimony point of view, we extended the model with random effects building on 2 classes, 3 classes and 4 classes solutions. 

The results of the random effect included models are presented in \textbf{Table \ref{tab:mixmodels}} under the Random effects model section. It is obvious that the BIC improves with the addition of the random effects which account for the nested structure of the data. Entropy indicates that 4 classes in individual level and 2 classes in the day level may be the best solution mathematically. However, after these solutions were checked in more details, the potentially most substantively interpretable model was found to be the 3$\times$3 random effect model, which is the model with 3 latent classes in the day level, and 3 latent classes in the individual level. We must emphasize that different researchers may have made decision slightly different from ours, we provided the descriptions and figures for other solutions in the \textbf{Appendix xxx} for reference. 

In the 3$\times$3 random effect model solution we have chosen, there were 39.5\%, 20.4\%, and 40.1\% observations classified into 3 latent groups in the day level. The overall counts and percentages for each responses within every time slot and the distributions of the solution are presented in \textbf{Table \ref{tab:daylevel}}. The trajectories illustrating the change of the probabilities of each response to carbohydrate eating during the hours of the day are shown separately by the three types of days in \textbf{Figure \ref{fig:level1}}.



\rowcolors{2}{gray!6}{white}

\begin{table}[H]
	
	\caption{\label{tab:mixmodels}Fit criteria for each model specification.}\vspace{-0.3cm}
	\centering
	\fontsize{9}{11}\selectfont
	\begin{tabular}[t]{lccccc}
		\hiderowcolors
		\toprule
		\multicolumn{1}{c}{ } & \multicolumn{5}{c}{\textbf{Number of day level classes}} \\
		\cmidrule(l{2pt}r{2pt}){2-6}
		\textbf{Model} & \textbf{1 class} & \textbf{2 classes} & \textbf{3 classes} & \textbf{4 classes} & \textbf{5 classes}\\
		\midrule
		\showrowcolors
		\addlinespace[0.3em]
		\multicolumn{6}{l}{\textbf{Fixed effects model}}\\
		\hspace{1em}No. of free parameters & 14 & 29 & 44 & 59 & 74\\
		\hspace{1em}\hspace{1em}Log-likelihood & -173793.306 & -172669.771 & -172039.204 & -171633.941 & -171377.292\\
		\hspace{1em}\hspace{1em}BIC & 347728.092 & 345632.608 & 344523.060 & 343864.121 & 343502.409\\
		\hspace{1em}\hspace{1em}Lo-Mendell-Rubun LRT & -- & $<$ 0.0001 & $<$ 0.0001 & $<$ 0.0001 & $<$ 0.0001\\
		\hspace{1em}\hspace{1em}Entropy & 1 & 0.310 & 0.392 & 0.510 & 0.481\\
		\addlinespace[0.3em]
		\multicolumn{6}{l}{\textbf{Random effects model}}\\
		\hspace{1em}2 individual level classes &  &  &  &  & \\
		\hspace{1em}\hspace{1em}No. of free parameters &  & 59 & 89 & 119 & \\
		\hspace{1em}\hspace{1em}Log-likelihood &  & -169331.132 & -168700.96 & -168366.193 & \\
		\hspace{1em}\hspace{1em}BIC &  & 339258.502 & 338301.338 & 337934.968 & \\
		\hspace{1em}\hspace{1em}Entropy &  & 0.581 & 0.569 & 0.555 & \\
		\hspace{1em}3 individual level classes &  &  &  &  & \\
		\hspace{1em}\hspace{1em}No. of free parameters &  & 89 & 134 & 179 & \\
		\hspace{1em}\hspace{1em}Log-likelihood &  & -166936.279 & -166348.815 & -166062.761 & \\
		\hspace{1em}\hspace{1em}BIC &  & 334771.968 & 334051.799 & 333934.448 & \\
		\hspace{1em}\hspace{1em}Entropy &  & 0.677 & 0.630 & 0.644 & \\
		\hspace{1em}4 individual level classes &  &  &  &  & \\
		\hspace{1em}\hspace{1em}No. of free parameters &  & 119 & 179 &  & \\
		\hspace{1em}\hspace{1em}Log-likelihood &  & -165441.731 & -164845.696 &  & \\
		\hspace{1em}\hspace{1em}BIC &  & 332086.045 & 331500.318 &  & \\
		\hspace{1em}\hspace{1em}Entropy &  & 0.729 & 0.659 &  & \\
		\bottomrule
		\multicolumn{6}{l}{{\scriptsize \textit{Note: }}}\\
		\multicolumn{6}{l}{{\scriptsize \textbf{Abbreviation}: No, number; BIC, Bayesian information criterion; Entropy, a pseudo-r-squared index;}}\\ 
		\multicolumn{6}{l}{{\scriptsize Lo-Mendel-Rubin LRT, likelihood ratio test comparing $q$ classes models with $q-1$ classes models.}}\\
	\end{tabular}
\end{table}

\rowcolors{2}{white}{white}
\vspace{-0.5cm}


Class 1 days \textbf{(Figure \ref{fig:level1}-A)} were given the name of "high carbohydrate day" since in these days of survey, the probabilities of carbohydrate contributed higher or equal to 50\% of the energy consumed were always higher than that in the other two types of days. Specifically, high carbohydrate days were characterised with probabilities of over 0.6 in time slots between 6 am to 9 am, 9 am to 12 am, and also 2 pm to 5 pm, during which the time slots may be interpreted as breakfast, morning snack, and afternoon snack time periods for many participants. Moreover, even during late night time period, such as 8 pm to 10 pm, and 10 pm to 6 am time slots, the probabilities of having higher carbohydrate contained food were still as high as 0.412, and 0.246, respectively.



\rowcolors{2}{gray!6}{white}

\begin{table}[H]
	
	\caption{\label{tab:daylevel}Day level latent class solution for three classes LCA model. (No individual level model)}\vspace{-0.3cm}
	\centering
	\fontsize{9}{11}\selectfont
	\begin{tabular}[t]{llccccc}
		\hiderowcolors
		\toprule
		\textbf{Time slots of} & \textbf{Responses to} & \multicolumn{1}{c}{ } & \multicolumn{1}{c}{ } & \textbf{\Centerstack{Class 1\\(39.5\%)}} & \textbf{\Centerstack{Class 2\\(20.4\%)}} & \textbf{\Centerstack{Class 3\\(40.1\%)}} \\
		\cmidrule(l{2pt}r{2pt}){5-5} \cmidrule(l{2pt}r{2pt}){6-6} \cmidrule(l{2pt}r{2pt}){7-7}
		 \textbf{the day} &  \textbf{carbohydrate intake} & $n$ & (\%) & \textbf{\Centerstack{High carbo- \\ hydrate day}} & \textbf{\Centerstack{Low carbo-\\hydrate day}} & \textbf{\Centerstack{Regular\\meals day}}\\
		\midrule
		\showrowcolors
		6 am – 9 am &  &  &  &  &  & \\
		& Not eating any food & 7655 & 31.2 & 0.129 & 0.450 & 0.320\\
		& Carbohydrate $<$ 50\%\textsuperscript{*} & 4500 & 18.4 & 0.130 & 0.267 & 0.128\\
		& Carbohydrate $\geqslant$ 50\%\textsuperscript{\dag} & 12328 & 50.4 & 0.741 & 0.283 & 0.552\\
		9 am – 12 am &  &  &  &  &  & \\
		& Not eating any food & 5447 & 22.2 & 0.237 & 0.079 & 0.401\\
		& Carbohydrate $<$ 50\% & 7227 & 29.5 & 0.158 & 0.492 & 0.173\\
		& Carbohydrate $\geqslant$ 50\% & 11809 & 48.2 & 0.605 & 0.429 & 0.426\\
		12 noon – 2 pm &  &  &  &  &  & \\
		& Not eating any food & 4783 & 19.5 & 0.156 & 0.356 & 0.019\\
		& Carbohydrate $<$ 50\% & 11112 & 45.4 & 0.405 & 0.413 & 0.560\\
		& Carbohydrate $\geqslant$ 50\% & 8588 & 35.1 & 0.439 & 0.231 & 0.421\\
		2 pm – 5 pm &  &  &  &  &  & \\
		& Not eating any food & 6926 & 28.3 & 0.130 & 0.123 & 0.659\\
		& Carbohydrate $<$ 50\% & 8277 & 33.8 & 0.249 & 0.602 & 0.076\\
		& Carbohydrate $\geqslant$ 50\% & 9280 & 37.9 & 0.621 & 0.276 & 0.266\\
		5 pm – 8 pm &  &  &  &  &  & \\
		& Not eating any food & 3043 & 12.4 & 0.114 & 0.199 & 0.034\\
		& Carbohydrate $<$ 50\% & 14240 & 58.2 & 0.516 & 0.590 & 0.639\\
		& Carbohydrate $\geqslant$ 50\% & 7200 & 29.4 & 0.370 & 0.211 & 0.328\\
		8 pm – 10 pm &  &  &  &  &  & \\
		& Not eating any food & 8722 & 35.6 & 0.322 & 0.291 & 0.480\\
		& Carbohydrate $<$ 50\% & 8898 & 36.3 & 0.266 & 0.551 & 0.212\\
		& Carbohydrate $\geqslant$ 50\% & 6863 & 28.0 & 0.412 & 0.158 & 0.308\\
		10 pm – 6 am &  &  &  &  &  & \\
		& Not eating any food & 16295 & 66.6 & 0.680 & 0.590 & 0.751\\
		& Carbohydrate $<$ 50\% & 4144 & 16.9 & 0.074 & 0.294 & 0.101\\
		& Carbohydrate $\geqslant$ 50\% & 4044 & 16.5 & 0.246 & 0.115 & 0.148\\
		\bottomrule
		\multicolumn{7}{l}{{\scriptsize \textit{Note: }}}\\
		\multicolumn{7}{l}{\scriptsize \textbf{Abbreviation:} LCA, latent class analysis}\\
		\multicolumn{7}{l}{{\scriptsize \textsuperscript{*} Carbohydrate $<$ 50\% indicates that within the time slot, carbohydrate contributed $<$ 50\% total energy intake.}}\\
		\multicolumn{7}{l}{{\scriptsize \textsuperscript{\dag} Carbohydrate $\geqslant$ 50\% indicates that within the time slot, carbohydrate contributed $\geqslant$ 50\% total energy intake.}}\\
		\multicolumn{7}{l}{}\\ 
	\end{tabular}
\end{table}

\rowcolors{2}{white}{white}
\vspace{-0.3cm}

Class 2 days \textbf{(Figure \ref{fig:level1}-B)} were named as "low carbohydrate day" because first of all, in these days the possibility of participants skipping breakfast was 0.45. And after 9 am, within low carbohydrate days, the probability of having food contained lower carbohydrate (contributed less than 50\% of total energy intake), was always higher than having higher carbohydrate contained food. In these days, participants also turned to have morning snacks (with only 0.079 possibility of not eating any food and similar probabilities of having either high or low carbohydrate contained food). This phenomenon may also be interpreted as having a long and late breakfast (brunch) in these mornings. The probability of not eating any food was the lowest for low carbohydrate days during the midnight time slot (10 pm to 6 am), with probability of 0.590 compared with 0.680 and 0.751 in the class 1 and class 3 days, respectively. 





\begin{figure}[H]
	%\vspace*{13cm}
	\centering
	\includegraphics[width=14cm]{Figures/level1.png}
	\decoRule
	\caption[Day Level Latent Class Solution.]{Day Level Latent Classes Solution.}
	\label{fig:level1}
\end{figure}
\vspace{-0.6cm}

Class 3 days \textbf{(Figure \ref{fig:level1}-C)} were called "regular meals day" due to the following reasons: 1) participants' dietary recordings showed that in these days there was almost 0 possibility of not eating any food at lunch (0.019 between 12 noon and 2 pm) and dinner (0.034 between 5 pm and 8 pm); 2) the probabilities of not eating during morning snack time (9 am to 12 am) and afternoon snack time (2 pm to 5 pm) were also the highest among the three types of days (0.401 and 0.659). 3) during these days, participants may have some high carbohydrate contained food between 8 pm and 10 pm (0.308), but the probability of not eating any food during 10 pm to 6 am next morning was 0.751, the highest among the three types of days. \vspace{-0.5cm}


\subsection{Features of the three carbohydrate eating time patterns}

The details of the characteristics of the three types of carbohydrate eating time pattern were listed in\textbf{ Table \ref{tab:day-level-features}}. Specifically, regular meals day turned to be recorded slightly more often in Northern Ireland, and Scotland. In terms of day of week distribution in the three types of days, there is strong evidence (\textit{p} < 0.001) that high carbohydrate days appeared more frequently in weekends (32.5\%) compared with low carbohydrate day (26.5\%) and regular meals day (30.7\%).

\rowcolors{2}{gray!6}{white}

\begin{table}
	
	\caption{\label{tab:day-level-features}Means (standard deviations), and counts (\%) of the characteristics of different types of days according to carbohydrate intake.}
	\centering
	\fontsize{9}{11}\selectfont
	\begin{tabular}[t]{lcccc}
		\hiderowcolors
		\toprule
		& \textbf{\Centerstack{High carbo-\\hydrate day}} & \textbf{\Centerstack{Low carbo-\\hydrate day}} & \textbf{\Centerstack{Regular\\meals day}} & \textbf{\textit{P} value}\textsuperscript{*}\\
		\midrule
		\showrowcolors
		Counts (\%) & 9667 (39.5) & 5002 (20.4\%) & 9814 (40.1\%) & \\
		Country (\%) &  &  &  & < 0.001\\
		\hspace{1em}England & 5627 (58.2) & 2972 (59.4) & 5291 (53.9) & \\
		\hspace{1em}Northern Ireland & 1194 (12.4) & 527 (10.5) & 1400 (14.3) & \\
		\hspace{1em}Scotland & 1527 (15.8) & 813 (16.3) & 1774 (18.1) & \\
		\hspace{1em}Wales & 1318 (13.6) & 690 (13.8) & 1349 (13.7) & \\
		Day of Week (\%) &  &  &  & < 0.001\\
		\hspace{1em}Monday & 1303 (13.5) & 715 (14.3) & 1370 (14.0) & \\
		\hspace{1em}Tuesday & 1266 (13.1) & 674 (13.5) & 1290 (13.1) & \\
		\hspace{1em}Wednesday & 1225 (12.7) & 740 (14.8) & 1233 (12.6) & \\
		\hspace{1em}Thursday & 1272 (13.2) & 752 (15.0) & 1425 (14.5) & \\
		\hspace{1em}Friday & 1458 (15.1) & 797 (15.9) & 1479 (15.1) & \\
		\hspace{1em}Saturday & 1537 (15.9) & 703 (14.1) & 1495 (15.2) & \\
		\hspace{1em}Sunday & 1605 (16.6) & 621 (12.4) & 1522 (15.5) & \\
		\hspace{1em}Weekend, Yes (\%) & 3142 (32.5) & 1324 (26.5) & 3017 (30.7) & < 0.001\\
		Total energy (kJ) & 7539.98 (2875.87) & 7160.22 (2922.15) & 7439.68 (2978.91) & < 0.001\\
		Carbohydrate (g) & 222.79 (89.84) & 209.70 (86.17) & 206.59 (84.42) & < 0.001\\
		Protein (g) & 71.36 (29.79) & 69.55 (30.20) & 73.29 (32.94) & < 0.001\\
		Fat (g) & 65.44 (33.27) & 63.94 (33.76) & 67.24 (34.73) & < 0.001\\
		Alcohol (g) & 11.76 (27.31) & 8.85 (24.25) & 13.80 (33.00) & < 0.001\\
		Total sugars (g) & 98.63 (56.03) & 88.03 (50.50) & 86.39 (50.96) & < 0.001\\
		Starch (g) & 124.07 (55.84) & 121.59 (56.13) & 120.11 (54.62) & < 0.001\\
		Non-milk extrinsic sugar\textsuperscript{\dag} & 59.45 (49.31) & 50.07 (43.41) & 50.41 (44.84) & < 0.001\\
		Fruit (g) & 107.40 (137.97) & 103.15 (129.08) & 92.76 (126.02) & < 0.001\\
		Yellow Red Green Vegetables (g) & 26.52 (46.44) & 26.84 (47.99) & 26.16 (45.99) & 0.681\\
		\bottomrule
		\multicolumn{5}{l}{{\scriptsize \textit{Note: }}}\\
		\multicolumn{5}{l}{{\scriptsize \textsuperscript{*} \textit{P} values were obtained from Pearson $\chi^2$ test for categorical variables, and one-way ANOVA comparing the means}}\\
		\multicolumn{5}{l}{{\scriptsize  in multiple groups for continuous variables;}}\\
		\multicolumn{5}{l}{{\scriptsize \textsuperscript{\dag} Non-milk extrinsic sugar is defined as: additionally added free sugar, such as table sugar, honey, glucose, fructose}}\\ 
		\multicolumn{5}{l}{{\scriptsize and glucose syrups, sugars added to food and sugars in fruit juices.}}\\
	\end{tabular}
\end{table}

\rowcolors{2}{white}{white}
\vspace{-0.5cm}


As expected, consumption of total energy (7539.98 kJ), total carbohydrate (222.79 g), total sugar (98.63 g), starch (124.07 g), and non-milk extrinsic sugar (59.45 g) were highest among high carbohydrate days (all \textit{p} < 0.001). On the other hand, the consumption of protein (73.29 g), total fat (67.24 g), and alcohol (13.80 g) were the highest in the so-called regular meals days. Moreover, in high carbohydrate days, participants turned to consume the highest amount of fruit (107.40 g). There was no evidence of any difference for the consumption of yellow, red, or green vegetables across the three types of days (\textit{p} = 0.681).

%---------------------------------
%	SUBSECTION 2
%-----------------------------------

\subsection{Individual level LCA solution}

In the random effect models we utilized the non-parametric approach, in which we added a level 2 (individual level) latent classes based on the random means from the level 1 (day level) latent class solution. The results of the individual level LCA solution for 2 and 3 classes are presented in \textbf{Figure \ref{fig:CB2level2}}, and \textbf{\ref{fig:level2}}. 

With two individual level latent classes \textbf{(Figure \ref{fig:CB2level2})}, one individual class is comprised of individuals with a relatively slightly higher proportion of having "low carbohydrate day" (22.1\%) compared to the other (17.4\%). This class represents nearly 65\% of the individuals. However, we believe these individual classes are not very distinguishable to each other.


\begin{figure}[H]
	%\vspace*{13cm}
	\centering
	\includegraphics[width=13cm]{Figures/CB2level2.png}
	\decoRule
	\caption[Multilevel Latent Class Solution ($3\times2$).]{Multilevel Latent Class Solution, 3 classes in day level, 2 classes in individual level.}
	\label{fig:CB2level2}
\end{figure}


With three individual level latent classes \textbf{(Figure \ref{fig:level2})}, a low-carbohydrate eaters class, a moderate-carbohydrate eaters class, and a high-carbohydrate eaters class emerges. 43.1\% participants were identified as high-carbohydrate eaters, in these individuals, about 50\% of the days (2 out of 4 days) of their dietary diary could be classified as having high carbohydrate days. Nearly 1 out of 4 days of their dietary diary were either "regular meals day" or "low carbohydrate day". 28.1\% participants fell into the low carbohydrate eaters class in the left hand side of \textbf{Figure \ref{fig:level2}}, their recordings of food intake showed that in more than 60\% of their days, they were having "regular meals" which was characterised as with highest amount of fat and alcohol consumptions as already described in \textbf{Table \ref{tab:day-level-features}}. Moderate carbohydrate eaters have comparable proportions (42.0\% vs. 40.0\%) of having high carbohydrate days and regular meals day, 18.0\% of their dietary diary were identified as low carbohydrate days.

\begin{figure}[H]
	%\vspace*{13cm}
	\centering
	\includegraphics[width=13cm]{Figures/level2.png}
	\decoRule
	\caption[Multilevel Latent Class Solution ($3\times3$).]{Multilevel Latent Class Solution, 3 classes in day level, 3 classes in individual level.}
	\label{fig:level2}
\end{figure}
\vspace{-0.6cm}



The social-demographic characteristics of the UK adults according to their individual level latent class membership are shown in \textbf{Table \ref{tab:Level2tab1}}. Moderate carbohydrate eaters were relatively younger (\textit{p} < 0.001), and slightly less from England (\textit{p} = 0.007). Gender distribution across the three types of carbohydrate eaters was fairly even (\textit{p} = 0.119). Distribution of the carbohydrate eater types turned out to be changing with the year of survey. Low carbohydrate eaters represented 32.5\% of the population in the first year of survey, but later dropped to lower than 30\% (lowest in the third year, 22.6\%) until the most recent year. Proportion of high carbohydrate eaters increased from 41.2\% to the highest (50.6\%) in the second year of the survey, but then started to decline to 38.4\% in the 8th year of survey (\textit{p} = 0.015). There was no evidence of difference in employment status across three types of carbohydrate eaters. However, strong evidence suggested that high carbohydrate eaters had the highest proportion (61.3\%) of living with a partner (\textit{p} < 0.001); moderate carbohydrate eaters had the lowest average income (27180.8 \textsterling/year), highest proportion of non-white population (20.5\%), and lower education level (23.3\% with degree of higher education) compared with either low or high carbohydrate eaters. 

Weighted means, percentages of anthropometric measurements, average of main nutrients intake, as well as biochemical characteristic profiles according to the latent carbohydrate eater groups are given in \textbf{Table \ref{tab:tab2}}. Low carbohydrate eaters had higher mean BMI (27.8 kg/m\textsuperscript{2}) and larger mean WC (98.9/89.9 cm in men/women) compared with 27.2, 27.3 kg/m\textsuperscript{2}, and 95.9/88.7 (men/women), 98.1/87.2 (men/women) cm in moderate and high carbohydrate eaters. Moderate carbohydrate eaters had the highest prevalence of being a current smoker (27.8\%), shortest time of daily physical activity (geometric mean: 0.87 hours/day), and the lowest prevalence of hypertension (20.2\%).

Average total energy intake over the 4 days of dietary survey was the highest (7985.8 kJ) in the low carbohydrate eaters group. As expected, the mean of total carbohydrate intake was 203.8 g, 218.3 g, and 233.4 g for low, moderate, and high carbohydrate eaters, respectively. Energy contribution from carbohydrate was close to 50\% in the high carbohydrate eaters, but was only 40.6\% in the low carbohydrate eaters. It is noteworthy that low carbohydrate eaters consumed the highest average amount of protein (79.9 g, 17.2\% of total energy), fat (74.7g, 35.4\% of total energy), and alcohol (20.8 g, 6.8\% of total energy). 

From the results of blood tests, 6.9\% of low carbohydrate eaters were found to be diabetic (diagnosed by A1C > 6.5\%), while the percentages of diabetes in the moderate and high carbohydrate eaters were 3.5\%, and 4.1\% (\textit{p} < 0.011), respectively. Although there was some evidence (\textit{p} = 0.027) that fasting blood glucose level may be slightly higher in non-diabetic low carbohydrate eaters, the geometric mean for A1C was probably lower in moderate carbohydrate eaters (4.72, 95\%CI: 5.39, 5.47). Cholesterol, HDL, and LDL were all lower in the moderate carbohydrate eaters, while no evidence of any difference of TG was found across three types of carbohydrate eaters. 



\rowcolors{2}{gray!6}{white} 

\begin{table}[H]

\caption{\label{tab:Level2tab1}Weighted means, percentages, and
95\% CIs of the social-demographic characteristics by carbohydrate eating latent class memberships in the UK
adults. \\ (NDNS RP 2008/09-2015/16, sample size = 6155)} \centering
\fontsize{9}{11}\selectfont

\begin{tabular}[t]{lcccc}
	\hiderowcolors
	\toprule
	\textbf{Variables} & \textbf{\Centerstack{Low carbo-\\hydrate eaters\\(n = 1730)}} & \textbf{\Centerstack{Moderate carbo-\\hydrate eaters\\(n = 1772)}} & \textbf{\Centerstack{High carbo-\\hydrate eaters\\(n = 2653)}} & \textbf{\textit{P} value} \textsuperscript{*}\\
	\midrule
	\showrowcolors
	Total (\%) & 28.4 (26.9, 29.9) & 28.7 (27.1, 30.3) & 43.0 (41.3, 44.7) & \\
	Country (\%) &  &  &  & 0.007\\
	\hspace{1em}England & 84.5 (81.7, 86.9) & 82.0 (79.3, 84.5) & 84.7 (82.3, 86.8) & \\
	\hspace{1em}Northern Ireland & 2.1 (1.6, 2.8) & 4.2 (3.2, 5.6) & 2.2 (1.7, 3.0) & \\
	\hspace{1em}Scotland & 9.1 (7.0, 11.8) & 8.6 (6.7, 11.1) & 8.0 (6.3, 10.2) & \\
	\hspace{1em}Wales & 4.3 (3.3, 5.6) & 5.1 (4.0, 6.4) & 5.1 (4.0, 6.4) & \\
	Age (years) & 51.0 (49.9, 52.1) & 40.3 (39.1, 41.6) & 51.7 (50.7, 52.7) & < 0.001\\
	Sex (\%) &  &  &  & 0.119\\
	\hspace{1em}Men & 50.0 (46.9, 53.1) & 50.2 (47.0, 53.5) & 46.6 (44.0, 49.1) & \\
	\hspace{1em}Women & 50.0 (46.9, 53.1) & 49.8 (46.5, 53.0) & 53.4 (50.9, 56.0) & \\
	Survey years (\% in rows) &  &  &  & 0.015\\
	\hspace{1em}1 & 32.5 (28.4, 36.9) & 26.3 (21.9, 31.2) & 41.2 (36.6, 46.0) & \\
	\hspace{1em}2 & 26.8 (22.6, 31.3) & 22.6 (18.6, 27.3) & 50.6 (45.8, 55.4) & \\
	\hspace{1em}3 & 22.6 (18.8, 26.9) & 33.7 (28.6, 39.2) & 43.6 (38.7, 48.7) & \\
	\hspace{1em}4 & 27.9 (24.1, 32.2) & 27.6 (23.8, 31.8) & 44.4 (40.2, 48.7) & \\
	\hspace{1em}5 & 27.9 (24.2, 32.0) & 28.7 (24.4, 33.5) & 43.3 (38.2, 48.6) & \\
	\hspace{1em}6 & 28.0 (24.0, 32.4) & 31.5 (26.9, 36.6) & 40.5 (35.8, 45.3) & \\
	\hspace{1em}7 & 29.1 (25.2, 33.4) & 29.0 (24.5, 34.0) & 41.8 (37.1, 46.7) & \\
	\hspace{1em}8 & 31.1 (27.3, 35.3) & 30.5 (25.9, 35.5) & 38.4 (34.1, 42.8) & \\
	Paid employment\textsuperscript{\dag} (\%) &  &  &  & 0.907\\
	\hspace{1em}Yes & 40.3 (37.0, 43.6) & 40.8 (37.1, 44.5) & 39.8 (37.1, 42.6) & \\
	\hspace{1em}No & 59.7 (56.4, 63.0) & 59.2 (55.5, 62.9) & 60.2 (57.4, 62.9) & \\
	Live with partner\textsuperscript{\ddag} (\%) &  &  &  & < 0.001\\
	\hspace{1em}Yes & 56.9 (53.6, 60.1) & 38.4 (35.2, 41.8) & 61.3 (58.7, 63.7) & \\
	\hspace{1em}No & 43.1 (39.9, 46.4) & 61.6 (58.2, 64.8)) & 38.7 (36.3, 41,3) & \\
	Household income, \textsterling/year & \Centerstack{36558.5\\(34800.2, 38316.8)} & \Centerstack{27180.8\\(25597.9, 28763.7)} & \Centerstack{32171.6\\(31024.9, 33318.2)} & < 0.001\\
	Ethnicity (\%) &  &  &  & \\
	\hspace{1em}White & 94.2 (92.4, 95.6) & 79.5 (76.4, 82.3) & 91.9 (90.1, 93.4) & < 0.001\\
	\hspace{1em}Non-White & 5.8 (4.4, 7.6) & 20.5 (17.7, 23.6) & 8.1 (6.6, 9.9) & \\
	Education (\%) &  &  &  & \\
	\hspace{1em}Degree or higher & 29.0 (26.1, 32.1) & 23.3 (20.5, 26.3) & 26.2 (24.1, 28.5) & 0.019\\
	\hspace{1em}Lower than degree & 71.0 (67.9, 73.9) & 76.7 (73.7, 79.5) & 73.8 (71.5, 75.9) & \\
	\bottomrule
	\multicolumn{5}{l}{{\scriptsize \textit{Note: }}}\\
	\multicolumn{5}{l}{{\scriptsize \textbf{Abbreviations}: CI, confidence intervals; NDNS RP, national dietary and nutrition survey rolling programme.}}\\
	\multicolumn{5}{l}{{\scriptsize Variables were weighted by individual weights.}}\\
	\multicolumn{5}{l}{{\scriptsize \textsuperscript{*} For continuous variables, the \textit{F} test was used to determine differences between latent classes with Bonferroni}}\\ 
	\multicolumn{5}{l}{{\scriptsize correction to account for multiple testing across $>$ 2 classes. For categorical variables, differences between}}\\
	\multicolumn{5}{l}{{\scriptsize  latent classes were assessed using the adjusted Pearson $\chi^2$ test for survey data.}} \\
	\multicolumn{5}{l}{{\scriptsize \textsuperscript{\dag} Paid employment was defined as being in paid employment during the last 4 weeks prior to the survey.} }\\
	\multicolumn{5}{l}{{\scriptsize \textsuperscript{\ddag} Live with partner was defined as either living with a married husband/wife or a legally recognised civil}} \\
	\multicolumn{5}{l}{{\scriptsize partnership.}}\\
\end{tabular}

\end{table} 
\rowcolors{2}{white}{white}





\rowcolors{2}{gray!6}{white}

\begin{table}[H]
	
	\caption{\label{tab:tab2}Weighted means, percentages, and 95\% CIs of the anthropometric measurements, average main nutrients intake and biochemical characteristics by carbohydrate eating latent class memberships in the UK adults. (NDNS RP 2008/09-2015/16, sample size = 6155)}
	\centering
	\fontsize{9}{11}\selectfont
	\begin{tabular}[t]{lcccc}
		\hiderowcolors
		\toprule
	\textbf{Variables} & \textbf{\Centerstack{Low carbo-\\hydrate eaters\\(n = 1730)}} & \textbf{\Centerstack{Moderate carbo-\\hydrate eaters\\(n = 1772)}} & \textbf{\Centerstack{High carbo-\\hydrate eaters\\(n = 2653)}} & \textbf{\textit{P} value} \textsuperscript{*}\\
		\midrule
		\showrowcolors
		BMI (kg/m\textsuperscript{2}) & 27.8 (27.4, 28.2) & 27.2 (26.7, 27.7) & 27.3 (26.9, 27.6) & 0.006\\
		WC (cm) &&&& \\ 
		\hspace{1em}Men &  98.9 (97.4, 100.5) & 95.9 (94.1, 97.8) &  98.1 (96.9, 99.2) & 0.056\\
		\hspace{1em}Women & 89.9 (88.7, 91.3) &  88.7 (87.1, 90.3) &  87.2 (86.1, 88.2) & 0.005\\
		Smoking status (\%) &  &  &  & \\
		\hspace{1em}Current & 20.4 (18.0, 23.0) & 27.8 (25.0, 30.9) & 17.1 (15.4, 19.0) & < 0.001\\
		\hspace{1em}Ex-smoker & 29.3 (26.5, 32.2) & 16.8 (14.6, 19.2) & 26.1 (24.9, 28.3) & \\
		\hspace{1em}Never & 50.3 (47.2, 32.2) & 55.4 (52.2, 58.6) & 56.8 (54.3, 59.3) & \\
		Physical activity (hours/day) \textsuperscript{\P} & 1.08 (0.97, 1.19) & 0.87 (0.77, 0.97)  & 1.07 (0.98, 1.16) & 0.005 \\
		Hypertension\textsuperscript{\dag}, Yes (\%) & 33.8 (30.2, 37.5) & 20.2 (17.0, 24.0) & 30.9 (26.9, 31.0) & < 0.001\\
		Total energy intake (kJ) & \Centerstack{7985.8\\(7823.3, 8146.3)} & \Centerstack{7341.8\\(7825.3, 8146.3)} & \Centerstack{7677.0\\(7555.8, 7799.8)} & < 0.001\\
		Carbohydrate intake (g) & \Centerstack{203.8\\(199.8, 207.8)} & \Centerstack{218.3\\(212.9, 223.7)} & \Centerstack{233.4\\(229.6, 237.2)} & < 0.001\\
		Carbohydrate percent\textsuperscript{\ddag} (\%) & 40.6 (40.2, 41.0) & 47.3 (46.8, 47.8) & 48.3 (47.9, 48.6) & < 0.001\\
		Protein intake (g) & 79.9 (77.9, 81.8) & 69.3 (67.6, 71.0) & 73.7 (72.5, 74.8) & < 0.001\\
		Protein percent (\%) & 17.2 (16.9, 17.5) & 16.3 (16.0, 16.6) & 16.5 (16.3, 16.6) & < 0.001\\
		Fat intake (g) & 74.7 (73.1, 76.4) & 63.8 (62.1, 65.5) & 65.7 (64.4, 67.0) & < 0.001\\
		Fat percent (\%) & 35.4 (34.9, 35.8) & 32.5 (32.1, 32.9) & 32.0 (31.7, 32.3) & < 0.001\\
		Alcohol intake (g) & 20.8 (18.3, 23.2) &  10.7 (9.4, 11.9) & 8.9 (8.1, 9.8) & < 0.001 \\
		Alcohol percent (\%) &  6.8 (6.3, 7.4) &  3.8 (3.4, 4.3) & 3.2 (2.9, 3.4)& < 0.001 \\
		Glucose (mmol/l) & 5.17 (5.12, 5.23) & 5.05 (4.99, 5.13) & 5.10 (5.05, 5.15) & 0.027\\
		A1C (\%) & 5.47 (5.44, 5.51) & 5.43 (5.39, 5.47) & 5.50 (5.48, 5.53) & 0.010\\
		DM \textsuperscript{\S} & 6.9 (5.0, 9.3) & 3.5 (2.3, 5.3) & 4.1 (2.9, 5.6) & 0.011\\
		TC (mmol/l) & 4.95 (4.84, 5.05) & 4.72 (4.62, 4.83) & 4.95 (4.87, 5.03) & 0.001 \\
		HDL (mmol/l) & 1.39 (1.35, 1.43)& 1.32 (1.28, 1.35) & 1.39 (1.36, 1.42)& 0.003 \\ 
		LDL (mmol/l) &  2.88 (2.79, 2.97) & 2.77 (2.68, 2.86) & 2.93 (2.86, 3.00)& 0.024 \\ 
		TG (mmol/l) &  1.14 (1.08, 1.19) &  1.11 (1.05, 1.17) & 1.10 (1.06, 1.15) & 0.629\\
		\bottomrule
%		\multicolumn{5}{l}{\textit{Note: }}\\
		\multicolumn{5}{l}{{\scriptsize \textbf{Abbreviations}: CI, confidence intervals; NDNS RP, national dietary and nutrition survey rolling programme;}}\\
		\multicolumn{5}{l}{{\scriptsize BMI body mass index; WC, waist circumference; A1C, haemoglobin A1c; DM, diabetes mellitus; TC, total cholesterol, }}\\
		\multicolumn{5}{l}{{\scriptsize HDL, high density lipoproteins; LDL, low density lipoproteins; TG, triglycerides.}} \\
		\multicolumn{5}{l}{{\scriptsize Glucose, A1C, TC, HDL, LDL, TG, and physical activity were expressed in geometric means (95\% CI) because }} \\
		\multicolumn{5}{l}{ the data were positively skewed.} \\
		\multicolumn{5}{l}{{\scriptsize Variables from the blood tests (glucose and A1C) were weighted by blood sample weights, the other variables were}}\\ 
		\multicolumn{5}{l}{{\scriptsize  weighted by nurse visiting weights. Glucose and A1C levels are estimated in subgroups of people without diabetes.}}\\
		\multicolumn{5}{l}{\textsuperscript{*} For continuous variables, the \textit{F} test was used to determine differences between latent classes with}\\ 
		\multicolumn{5}{l}{{\scriptsize  Bonferroni correction to account for multiple testing across $>$ 2 classes. For categorical variables, differences}}\\
		\multicolumn{5}{l}{{\scriptsize  between latent classes were assessed using the adjusted Pearson $\chi^2$ test for survey data.}}\\ 
		\multicolumn{5}{l}{{\scriptsize \textsuperscript{\P} Physical activity was calculated as mean time spent at moderate or vigorous physical activity including both }}\\
		\multicolumn{5}{l}{{\scriptsize work-related and recreational activities during the most recent month before the survey.}}\\
		
		\multicolumn{5}{l}{{\scriptsize \textsuperscript{\dag} Hypertension was defined as either systolic blood pressure $\geqslant$ 140 mmHg or diastolic blood pressure $\geqslant$ 90 mmHg,}}\\
		\multicolumn{5}{l}{{\scriptsize or under treatment for hypertension.}}\\
		\multicolumn{5}{l}{{\scriptsize \textsuperscript{\ddag} Carbohydrate percent indicates the percentage of energy from carbohydrate in total energy intake.}}\\  
		\multicolumn{5}{l}{{\scriptsize \textsuperscript{\S} DM was defined by A1C $>$ 6.5\%.}}\\
	\end{tabular}
\end{table}

\rowcolors{2}{white}{white}

%----------------------------------------------------------------------------------------
%	SECTION 2
%----------------------------------------------------------------------------------------

\subsection{Association between individual level latent classes and hypertension, obesity, and diabetes.}



\begin{sidewaystable}
\rowcolors{2}{gray!6}{white}
%\begin{table}
	\caption{\label{tab:tab1hypetension}Weighted means, percentages, and 95 \% CIs of the characteristics by hypertension status in the UK adults. \\(NDNS RP 2008/09-2015/16, sample size = 6155)}
	\centering
	\fontsize{9}{11}\selectfont
	\begin{tabular}[t]{lcccccc}
		\hiderowcolors
		\toprule
		\multicolumn{1}{c}{ } & \multicolumn{3}{c}{\textbf{Men (n = 2537)}} & \multicolumn{3}{c}{\textbf{Women (n = 3618)}} \\
		\cmidrule(l{2pt}r{2pt}){2-4} \cmidrule(l{2pt}r{2pt}){5-7}
		& \textbf{Non-hypertensive} & \textbf{Hypertensive} & \textbf{\textit{P} value\textsuperscript{*}} & \textbf{Non-hypertensive} & \textbf{Hypertensive} & \textbf{\textit{P} value\textsuperscript{*}}\\
		\midrule
		\showrowcolors
		Weighted prevalence (\%) & 69.6 (66.6, 72.5)  & 30.4 (27.5, 33.4) &  & 72.5 (69.8, 75.0) & 27.5 (25.0, 30.2) & \\
		Age (years) & 43.2 (41.7, 44.7) & 59.9 (58.0, 61.7) & < 0.001 & 43.9 (42.7, 45.1) & 64.9 (63.4, 66.5) & < 0.001\\
		Country (\%) &  &  & 0.109 &  &  & 0.631\\
		\hspace{1em}England & 84.7 (80.9, 87.2) & 85.4 (81.2, 88.8) &  & 84.0 (81.0, 86.6) & 83.5 (79.1, 87.0) & \\
		\hspace{1em}Northern Ireland & 3.3 (2.2, 4.8) & 1.6 (0.8, 3.1) &  & 2.5 (1.9, 3.5) & 2.6 (1.5, 4.3) & \\
		\hspace{1em}Scotland & 8.6 (6.3, 11.7) & 7.1 (4.6, 10.9) &  & 8.7 (6.5, 11.7) & 7.9 (5.1, 11.8) & \\
		\hspace{1em}Wales & 3.9 (2.7, 5.6) & 5.9 (4.0, 8.5) &  & 4.7 (3.7, 6.0) & 6.1 (4.3, 8.6) & \\
		Ethnicity (\%) &  &  & 0.534 &  &  & 0.126\\
		\hspace{1em}White & 89.6 (86.5, 92.0) & 91.1 (86.2, 94.4) &  & 85.7 (82.7, 88.3) & 90.2 (85.0, 93.7) & \\
		\hspace{1em}Non-white & 10.4 (8.0, 13.5) & 8.9 (5.6, 13.8) &  & 14.3 (11.7, 17.3) & 9.8 (6.3, 15.0) & \\
		Education (\%) &  &  & 0.006 &  &  & < 0.001\\
		\hspace{1em}Degree or higher & 30.3 (26.6, 34.2) & 21.5 (17.3, 26.5) &  & 33.0 (29.9, 36.3) & 19.7 (15.8, 24.3) & \\
		\hspace{1em}Lower than Degree & 69.7 (65.8, 73.4) & 78.5 (73.5, 82.7) &  & 67.0 (63.7, 70.1) & 80.3 (75.7, 84.2) & \\
		Household income,\textsterling/year & 34006.5 (31972.9, 36040.1) & 32280.5 (29875.6, 34685.4) & 0.284 &  32741.5 (31009.9, 34473.1) & 27862.0 (25557.0, 30167.0)  &  < 0.001  \\
		Live with partner\textsuperscript{\ddag}, Yes, (\%) & 56.1 (51.8, 61.4) & 66.6 (61.3, 71.5) & 0.002   & 48.7 (45.1, 52.3) & 58.9 (53.6, 63.9)  & 0.002    \\
		Smoking status &  &  & < 0.001 &  &  & < 0.001\\
		\hspace{1em}Current & 19.7 (16.6, 23.1) & 12.9 (9.5, 17.2) &  & 15.2 (13.1, 17.6) & 8.5 (6.2, 11.6) & \\
		\hspace{1em}Ex-smoker & 24.2 (21.1, 27.6) & 38.8 (33.4, 44.5) &  & 21.6 (19.1, 24.4) & 32.2 (27.3, 37.4) & \\
		\hspace{1em}Never & 56.2 (52.1, 60.1) & 48.3 (42.7, 54.0) &  & 63.2 (60.1, 66.2) & 59.3 (54.0, 64.4) & \\
		Physical
		activity (hours/day) \textsuperscript{\dag} & 1.52 (1.33, 1.72) & 1.29 (1.08, 1.53) & 0.134 & 0.81 (0.73, 0.89) & 0.53 (0.42, 0.64) & < 0.001\\
		BMI (kg/m\textsuperscript{2}) & 26.8 (26.4, 27.2) & 29.5 (28.9, 29.9) & < 0.001 & 26.4 (26.1, 26.8) & 29.8 (29.2, 30.5) & < 0.001\\
		WC (cm) & 95.0 (93.9, 96.2) & 104.6 (103.2, 106.1) & < 0.001 & 85.7 (84.8, 86.6) & 95.7 (94.2, 97.2) & < 0.001\\
		DM\textsuperscript{\S} (\%) & 3.7 (2.4, 5.7) & 12.6 (8.9, 17.5) & < 0.001 & 1.8 (1.0, 3.3) & 7.9 (5.1, 11.9) & < 0.001 \\ 
		Carbohydrate eating patterns (\%) &  &  & < 0.001 &  &  & < 0.001\\
		\hspace{1em}Low & 28.3 (24.8, 32.2) & 37.1 (32.0, 42.5) &  & 26.9 (24.1, 29.9) & 32.0 (27.2, 37.2) & \\
		\hspace{1em}Moderate & 30.8 (26.9, 35.0) & 19.3 (15.3, 24.1) &  & 29.6 (26.4, 33.0) & 18.4 (14.5, 22.9) & \\
		\hspace{1em}High & 40.8 (36.9, 44.9) & 43.6 (38.2, 49.2) &  & 43.5 (40.3, 46.8) & 49.7 (44.1, 55.2) & \\
		Total energy intake (kJ) & 9021.4 (8791.9, 9251.0) & 8366.2 (8094.9, 8637.4) & < 0.001 & 6802.6 (6681.1, 6924.0) & 6396.7 (6217.1, 6576.2) & < 0.001\\
		Carbohydrate intake (g) & 259.2 (252.9, 265.3) & 235.3 (227.8, 242.8) & < 0.001 & 198.0 (194.2, 201.8) & 184.5 (178.8, 190.1) & < 0.001\\
		\bottomrule
		\multicolumn{7}{l}{\textit{Note: }}\\
		\multicolumn{7}{l}{\textbf{Abbreviations}: CI, confidence intervals; NDNS RP, national dietary and nutrition survey rolling programme; BMI body mass index; WC, waist circumference.}\\
		\multicolumn{7}{l}{Variables are weighted by nurse visiting weights.}\\
		\multicolumn{7}{l}{\textsuperscript{*} Significant sex-specific differences by hypertension status assessed using an \textit{F} test for continuous variables or design-adjusted Pearson $\chi^2$ test.}\\
		\multicolumn{7}{l}{\textsuperscript{\ddag} Live with partner was defined as either living with a married husband/wife or a legally recognised civil partnership.}\\
		\multicolumn{7}{l}{\textsuperscript{\dag} Physical activity was calculated as mean time spent at moderate or vigorous physical activity including both work-related and recreational activities.}\\
		\multicolumn{7}{l}{\textsuperscript{\S}  DM was defined by A1C $>$ 6.5\%.}
	\end{tabular}
\rowcolors{2}{white}{white}
\end{sidewaystable}




\begin{sidewaystable}
	\rowcolors{2}{gray!6}{white}
	
%	\begin{table}
		
		\caption{\label{tab:tab1BMI}Weighted means, percentages, and 95 \% CIs of the characteristics by BMI categories in the UK adults. \\ (NDNS RP 2008/09-2015/16, sample size = 6155)}
		\centering
		\fontsize{8}{11}\selectfont
		\begin{tabular}[t]{lcccccccc}
			\hiderowcolors
			\toprule
			\multicolumn{1}{c}{ } & \multicolumn{4}{c}{\textbf{Men (n = 2537)}} & \multicolumn{4}{c}{\textbf{Women (n = 3618)}} \\
			\cmidrule(l{2pt}r{2pt}){2-5} \cmidrule(l{2pt}r{2pt}){6-9}
			& \textbf{Normal weight} & \textbf{Overweight} & \textbf{Obese} & \textbf{\textit{P} value\textsuperscript{*}} & \textbf{Normal weight} & \textbf{Overweight} & \textbf{Obese} & \textbf{\textit{P} value\textsuperscript{*}}\\
			\midrule
			\showrowcolors
			Weighted prevalence (\%) & 30.9 (28.0, 33.9) & 43.4 (40.4, 46.4) & 25.7 (23.2, 28.4) &  & 41.7 (39.0, 44.4) & 30.9 (28.4, 33.5) & 27.4 (25.1, 29.9) & \\
			BMI (kg/m\textsuperscript{2}) & 22.6 (22.3, 22.8) & 27.3 (27.2, 27.5) & 33.7 (33.3, 34.2) & < 0.001 & 22.2 (22.0, 22.4) & 27.3 (27.2, 27.5) & 35.0 (34.6, 35.4) & < 0.001\\
			WC (cm) & 84.5 (83.6, 85.4) & 97.1 (96.4, 97.8) & 112.7 (111.6, 113.9) & < 0.001 & 76.9 (76.2, 77.5) & 89.0 (88.3, 89.7) & 103.7 (102.6, 104.7) & < 0.001\\
			Age (years) & 40.3 (38.2, 42.4) & 49.6 (47.9, 51.2) & 50.4 (48.5, 52.3) & < 0.001 & 45.0 (43.4, 46.7) & 50.4 (48.6, 52.3) & 50.9 (49.1, 52.7) & < 0.001\\
			Country (\%) &  &  &  & 0.236 &  &  &  & 0.589\\
			\hspace{1em}England & 83.9 (79.2, 87.7) & 86.9 (83.6, 89.7) & 81.6 (76.7, 85.7) &  & 84.3 (80.4, 87.5) & 83.7 (79.7, 87.0) & 82.4 (78.2, 85.9) & \\
			\hspace{1em}Northern Ireland & 3.3 (1.9, 5.8) & 2.3 (1.4, 3.6) & 3.2 (2.0, 4.9) &  & 3.0 (2.1, 4.2) & 2.2 (1.5, 3.2) & 3.4 (2.3, 5.0) & \\
			\hspace{1em}Scotland & 9.1 (6.1, 13.5) & 6.5 (4.4, 9.7) & 9.0 (5.9, 13.3) &  & 9.0 (6.1, 13.0) & 9.0 (6.3, 12.8) & 8.5 (5.7, 12.6) & \\
			\hspace{1em}Wales & 3.7 (2.3, 5.8) & 4.3 (3.1, 5.9) & 6.3 (4.0, 9.7) &  & 3.8 (2.9, 5.1) & 5.1 (3.6, 7.2) & 5.7 (4.2, 7.7) & \\
			Ethnicity (\%) &  &  &  & 0.466 &  &  &  & 0.879\\
			\hspace{1em}White & 88.7 (83.9, 92.2) & 89.1 (85.6, 91.9) & 91.9 (87.3, 94.9) &  & 88.4 (84.9, 91.19 & 88.6 (84.5, 91.7) & 87.3 (82.5, 90.9) & \\
			\hspace{1em}Non-white & 11.3 (7.8, 16.1) & 10.9 (8.1, 14.4) & 8.1 (5.1, 12.7) &  & 11.6 (8.9, 15.1) & 11.4 (8.3, 15.5) & 12.7 (9.1, 17.5) & \\
			Education (\%) &  &  &  & 0.022 &  &  &  & < 0.001\\
			\hspace{1em}Degree or higher & 29.5 (24.5, 35.0) & 28.3 (24.3, 32.7) & 20.1 (16.0, 25.0) &  & 35.7 (31.8, 39.8) & 24.2 (20.4, 28.4) & 19.4 (16.1, 23.2) & \\
			\hspace{1em}Lower than Degree & 70.5 (65.0, 75.5) & 71.7 (67.3, 75.7) & 79.9 (75.0, 84.0) &  & 64.3 (60.2, 68.2) & 75.8 (71.6, 79.6) & 80.6 (76.8, 83.9) & \\
            Household income, \textsterling/year & \Centerstack{33695.9 \\ (30462.3, 36929.7)} & \Centerstack{35059.6 \\ (32949.7, 37169.5)} & \Centerstack{30295.5 \\ (27948.5, 32642.6)} & 0.011 & \Centerstack{34594.1 \\ (32326.8, 36861.4)} & \Centerstack{29777.7 \\ (27808.3, 31747.2)} & \Centerstack{27230.6\\ (25259.1, 29202.0)} & < 0.001 \\
            Live with partner\textsuperscript{\ddag}, Yes, (\%) & 40.3 (34.8, 46.1) & 65.3 (60.8, 69.6) & 65.6 (60.1, 70.8) & < 0.001 & 47.6 (43.2, 52.1) & 52.2 (47.5, 57.0) & 51.7 (46.7, 56.6) & 0.288 \\
			Smoking status &  &  &  & < 0.001 &  &  &  & 0.042\\
			\hspace{1em}Current  & 32.0  (26.8, 37.7) & 18.7 (15.5, 22.4) & 19.2 (15.0, 24.3) &  & 19.5 (16.4, 22.9) & 17.8 (14.8, 21.4) & 16.4 (13.1, 20.3) & \\
			\hspace{1em}Ex-smoker & 17.3 (13.5, 22.0) & 28.6 (24.8, 32.7) & 32.9 (27.9, 38.4) &  & 19.0 (15.9, 22.5) & 24.4 (20.8, 28.3) & 26.9 (22.8, 31.6) & \\
			\hspace{1em}Never & 50.6 (44.8, 56.4) & 52.7 (48.2, 57.1) & 47.9 (42.1, 53.7) &  & 61.6 (57.4, 65.5) & 57.8 (53.3, 62.2) & 56.7 (51.8, 61.4) & \\
			Physical activity\textsuperscript{\dag} (hours/day) & 1.58 (1.33, 1.85) & 1.42 (1.24, 1.62) & 1.41 (1.15, 1.70) & 0.547 & 0.84 (0.74, 0.94) & 0.71 (0.62, 0.79) & 0.65 (0.53, 0.78) & 0.038\\
			Carbohydrate eating patterns (\%) &  &  &  & 0.072 &  &  &  & 0.253\\
			\hspace{1em}Low & 25.9 (21.0, 31.5) & 30.6 (26.6, 35.0) & 31.4 (26.6, 36.6) &  & 24.8 (21.5, 28.5) & 26.8 (22.8, 31.2) & 29.5 (25.3, 34.1) & \\
			\hspace{1em}Moderate & 34.2 (28.6, 40.4) & 25.5 (21.9, 29.6) & 25.5 (20.6, 31.0) &  & 27.6 (23.8, 31.8) & 26.3 (22.3, 30.8) & 29.8 (25.4, 34.6) & \\
			\hspace{1em}High & 39.9 (34.2, 45.8) & 43.8 (39.6, 48.2) & 43.1 (37.7, 48.7) &  & 47.6 (43.3, 51.9) & 46.9 (42.4, 51.4) & 40.7 (36.0, 45.6) & \\
			Total energy intake (kJ) & \Centerstack{9351.2 \\ (8961.7, 9740.7)} & \Centerstack{8786.9 \\ (8595.1, 8978.7)} & \Centerstack{8465.3 \\ (8196.4, 8734.1)} & 0.001 & \Centerstack{7048.9 \\ (6894.4, 7203.4)} & \Centerstack{6570.1 \\(6406.2, 6734.0)} & \Centerstack{6566.4 \\ (6360.7, 6772.1)} & < 0.001\\
			Carbohydrate intake (g) & \Centerstack{268.7 \\ (258.3, 279.2)} & \Centerstack{250.1\\ (244.1, 256.1)} & \Centerstack{239.1 \\ (231.3, 246.8)} & < 0.001 & \Centerstack{205.8 \\(200.2, 211.3)} & \Centerstack{190.1 \\ (185.3, 194.9)} & \Centerstack{189.8 \\ (183.7, 195.9)} & < 0.001\\
			\bottomrule
			\multicolumn{9}{l}{\textit{Note: }}\\
			\multicolumn{9}{l}{\textbf{Abbreviations}: CI, confidence intervals; NDNS RP, national dietary and nutrition survey rolling programme; BMI body mass index; WC, waist circumference.}\\
			\multicolumn{9}{l}{Variables are weighted by nurse visiting weights.}\\
			\multicolumn{9}{l}{\textsuperscript{*} Significant sex-specific differences by BMI categories assessed using an \textit{F} test (with Bonferroni correction to account for multiple testing across $>$ 2 groups)} \\ 
			\multicolumn{9}{l}{for continuous variables or design-adjusted Pearson $\chi^2$ test for categorical variables}\\
			\multicolumn{9}{l}{\textsuperscript{\ddag} Live with partner was defined as either living with a married husband/wife or a legally recognised civil partnership.}\\
			\multicolumn{9}{l}{\textsuperscript{\dag} Physical activity was calculated as mean time spent at moderate or vigorous physical activity including both work-related and recreational activities.}\\
		\end{tabular}
%	\end{table}
	
	\rowcolors{2}{white}{white}
\end{sidewaystable}





%
%
%
%\section{Association between individual level latent classes and obesity}
%
%
%
%\section{Association between individual level latent classes and diabetes}