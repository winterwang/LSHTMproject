% Chapter Template

\chapter{Discussion and Conclusion} % Main chapter title

\label{Chapter 4} % Change X to a consecutive number; for referencing this chapter elsewhere, use \ref{ChapterX}





%----------------------------------------------------------------------------------------
%	SECTION 1
%----------------------------------------------------------------------------------------

\section{Main Findings}

\subsection{Carbohydrate eating patterns}

Using multilevel LCA as a novel technique, and the NDNS RP dietary database, this project examined carbohydrate eating temporal patterns firstly in the day level, based on which, individual level carbohydrate eating patterns were also defined subsequently. 

Among the dietary diaries collected, there were three distinct latent classes specifically for carbohydrate intake: 1) high probabilities of having high carbohydrate contained food across the hours of day (high percentage carbohydrate day); 2) low carbohydrate food dominant through out the hours of day (low percentage carbohydrate day); and 3) always having lunch and dinner day (regular meals day). And from these day level classification and their features, one might anticipate that individuals who managed to follow the regular meals day might be eating a relatively healthier diet because it seemed to be a regular temporal eating habit; at this time point, we also believe that those who follow either high or low percentages would probably consume total energy. 

However, when the MLCA extended the model to individual level, three types of persons are further defined depending on their 4-day-diary: 1) low carbohydrate eaters, who mostly followed "regular meals day"; 2) moderate carbohydrate eaters, who had similar probabilities of following either "regular meals day" or "high percentage carbohydrate day"; 3) high carbohydrate eaters, who followed "high percentage carbohydrate day" for half of their survey. For the first time, as far as we know, the day-to-day food intake pattern variation within individuals was successfully captured. Results from the MLCA models showed that from the perspective of carbohydrate consumption, people were indeed switching between different types of carbohydrate eating types even within a short term period of survey. The MLCA models allowed the probability of a certain type of carbohydrate eating day to vary across individuals. This properly accounted for the fact that for some people, their probability of following a type of food eating pattern during the survey was higher/lower than that in the others. Apparently, assuming that a person is always following or characterised by one type of food eating pattern is not appropriate.

Surprisingly, low carbohydrate eaters turned out to consume the highest amount of total energy, 




\subsection{Association between carbohydrate eating patterns and health outcomes}








%----------------------------------------------------------------------------------------
%	SECTION 2
%----------------------------------------------------------------------------------------




\section{Strengths and limitations}



\begin{itemize}
	\item MLCA ignored the order of observation days.
	\item We used the maximum probability rule and ignored that these are just probabilities.
\end{itemize}



\section{Conclusions}