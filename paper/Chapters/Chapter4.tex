% Chapter Template

\chapter{Discussion and Conclusion} % Main chapter title

\label{Chapter 4} % Change X to a consecutive number; for referencing this chapter elsewhere, use \ref{ChapterX}





%----------------------------------------------------------------------------------------
%	SECTION 1
%----------------------------------------------------------------------------------------

%\section{Main Findings}

\section{Carbohydrate eating patterns}

Using multilevel LCA as a novel technique, and data from the NDNS RP, this project examined carbohydrate temporal eating patterns firstly at the day level, based on which, individual level carbohydrate eating patterns were also defined subsequently. 

Among the dietary diaries collected, there were three distinct latent classes specifically for carbohydrate intake: 1) high probabilities of having high carbohydrate contained food across the hours of day (high percentage carbohydrate day); 2) low carbohydrate food dominant throughout the hours of day (low percentage carbohydrate day); and 3) always having lunch and dinner day (regular meals day). From these day level classifications and their features, one might anticipate that individuals followed much of class 3) days, the regular meals day, might be eating a healthier diet because of regular eating habits. We might also speculate that those who followed either high or low carbohydrate percentage days would consume higher total energy than those who followed mostly regular meals days. 

However, when the MLCA extended the model to an individual level, three types of persons were further defined depending on their 4-day-diary: 1) low carbohydrate eaters, who mostly followed "regular meals day"; 2) moderate carbohydrate eaters, who had similar probabilities of following either "regular meals day" or "high percentage carbohydrate day"; 3) high carbohydrate eaters, who followed "high percentage carbohydrate day" for half of their survey. For the first time, as far as we know, the day-to-day food intake pattern variation within individuals was successfully captured by MLCA models. Results from the MLCA models showed that from the perspective of carbohydrate consumption, people were indeed changing their diet from day to day even over the captured four days. The MLCA models allowed the probability of following a certain type of carbohydrate eating day to vary across individuals. This properly accounted for the fact that for some people, their probability of following a type of food eating pattern during the survey could be higher/lower than that in the others. This finding also suggested that assuming a person will always follow a certain type of temporal eating pattern is not appropriate.

Surprisingly, low carbohydrate eaters whose dietary recordings suggested that they were mostly following a regular temporal meals pattern turned out to be consuming the highest amount of total energy among the three types of carbohydrate eaters. Detailed profiling of energy composition according to the time slots revealed that low carbohydrate eaters had a higher proportion of energy contributed by both alcohol and fat across the 7 time slots. A high percentage of fat consumption was shown in all 7-time slots, energy coming from alcohol exceeded more than one-fifth of the total energy after 8 pm. These findings explained why they were actually consuming the highest energy among the three types of carbohydrate eaters. In the meantime, we also found that participants consuming low carbohydrate food had a higher prevalence of diabetes, hypertension, and obesity. These health issues might lead them to replace carbohydrates in their diet with other energy sources such as fat, protein, or even alcohol. Therefore, there is a possibility that they chose to follow low carbohydrate diets for health purposes, but many of them were replacing carbohydrates with higher energy condensed food or even alcohol at night which might indeed be a public health concern. 

Next, when looking into the details of the timing and composition of the energy intake among the moderate carbohydrate eaters, we realised that although these individuals did not consume as much alcohol as low carbohydrate eaters did at night, they consumed the highest amount of energy, especially during the time period as late as after 10 pm. People who fell into moderate carbohydrate eaters group seemed to have the tendency of having their food or meals later than high carbohydrate eaters. They consumed the highest amount of carbohydrates and also total energy among three types of carbohydrate eaters during the following time slots: 9-12 noon, 2-5 pm, 8-10 pm, and 10pm-6am. These individuals in the NDNS RP were younger, mostly single, with lower average income, and lower education level. They might correspond to the "late eaters" defined by previous studies \parencite{leech2017temporal, mansukhani2018investigating}. 

Finally, the high carbohydrate eaters identified by our MLCA models had the highest absolute total amount of carbohydrate intake. Most (nearly 70\%) of their total energy intake occurred during 6-9 am, 12-2 pm, and 5-8 pm time slots although their total energy consumption was not the highest (between low and moderate carbohydrate eaters). High carbohydrate eaters were also found to be the ones that consumed the least amount of energy after 8 pm. Therefore, contrary to what was anticipated at the beginning, people who followed high percentage carbohydrate days for most of their time were potentially eating a healthier diet compared with the other two eating patterns. 

Our analyses looking for different temporal carbohydrate eating patterns also highlighted the complexity of eating behaviours in the population and the utility of exploratory, data-driven methods to objectively identify eating patterns that reflect both timing and quantities of food intake, which may not have been detected so far in the literature.


\section{Associations between carbohydrate eating patterns and health outcomes}

Men who were classified as moderate carbohydrate eaters were estimated to have lower odds of having hypertension compared with low carbohydrate eaters, after adjustment for age, relationship status (living with a partner or not), educational level, BMI, smoking status and total energy intake. As discussed above, moderate carbohydrate eaters tended to have meals (or energy intake) later in time compared with high carbohydrate eaters. But, it is noteworthy that low carbohydrate eaters also obtained a large amount of energy from both fat and alcohol at night. Therefore, considering that moderate carbohydrate eaters were younger than low carbohydrate eaters (although age was adjusted in the full models), it is probable that reverse causality exists here (also due to the nature of cross-sectional study). That is, they were potentially both late eaters, however, with increased age (and so with increased health-related problems/concerns) some of the individuals modified their habits,  replacing carbohydrate food with other energy sources (so that they became low carbohydrate eaters) leading to the phenomenon of lower odds of hypertension in moderate carbohydrate eaters. This phenomenon may be supported by some current trend in following low carbohydrate diets (ketogenic diets) which were found actually with limited weight losing effect and even associated with adverse events such as nonalcoholic fatty liver disease (NAFLD), and insulin resistance \parencite{kosinski2017effects}. Although these hypotheses cannot be determined by the cross-sectional data from NDNS RP, if they are true, the energy sources individuals used to replace carbohydrate in their diet were apparently not wisely chosen. 

Among women, relationship status acted as an interaction factor influencing the associations between carbohydrate eating patterns and BMI and abdominal obesity (WC). Directions of the associations were opposite to each other depending on whether women were living with a partner or not. This interaction effect was more noticeable when looking at abdominal obesity measurement. High carbohydrate eaters who lived with partners had lower BMI and WC than low carbohydrate eaters, while moderate carbohydrate eaters who lived alone had higher WC than their low carbohydrate eating counterparts, after adjustment of age, education level, smoking status, total energy intake, and alcohol consumption. High carbohydrate eaters who were characterised by high and early daily energy consumption and low fat and alcohol intake may reflect a healthier diet and lifestyle, but this might be different between women who lived alone and those who lived with their partners. It was often assumed that living alone may be associated with a lower diversity of food intake and a higher likelihood of having unhealthy dietary choices \parencite{hanna2015relationship}. Therefore, there may be differences in the actual contents consumed in their high carbohydrates, or there may be other social, psychological or lifestyle-related factors related with living alone which were not measured or were not included in the models. Since the inverse association between high carbohydrate eating pattern and BMI or abdominal obesity were only observed among women who lived with partners, further investigation of this hypothesis is needed. In addition, the reason why moderate carbohydrate eaters' WC was larger than low carbohydrate eaters only among women who lived alone is unknown. Given that the evidence of this association was weak and borderline significant, whether it was just a false positive finding should also be explored in other studies.

%but if the reverse causality reason still applies, women who lived alone may be less likely to modify their diet (replacing carbohydrate with fat or alcohol) and therefore they stayed 




%Women lived with partners had higher proportion of being classified as high carbohydrate eaters 



%----------------------------------------------------------------------------------------
%	SECTION 2
%----------------------------------------------------------------------------------------




\section{Limitations and strengths}
 
Several limitations in the current project merit consideration.  

First, we ignored the order of observation days in the MLCA models. The food consumption diaries were accomplished by participants for at least 3 out of 4 consecutive days. In the multilevel analyses, these 3 or 4 days' observations were treated as if they were exchangeable in the models. Since one's diet might change according to the season, day of the week (weekdays or weekends), or sometimes depending on what one had consumed the day before, exchangeability of daily diaries is a strong assumption and cannot be overcome by the MLCA models adopted here. Other statistical techniques which could take the order or the longitudinal nature of the data into account, such as repeated measures latent class analysis (RMLCA), latent transition analysis (LTA)\parencite{collins2010latent}, or latent class growth analysis (LCGA) \parencite{davidian2008growth,jung2008introduction,andruff2009latent} are not applicable for the NDNS RP dataset. The NDNS RP participants were allowed to choose which days to begin their food diaries, but the alternative models above would require that the 3 or 4 repeated measurements of food diaries be recorded at the same time points longitudinally. Specifically, LCGA models (also called growth mixture model) will also need to model the change of the odds of the probability over time as a function (quadratic, or cubic) which is apparently not the objective in the current project.


Second, the classification of individuals to latent carbohydrate eating classes was defined by maximum posterior probability assignment rule. This approach assigns individuals to the class for which they have the highest posterior probability of membership \parencite{nagin2005group}. The other approach is multiple pseudo-class draws \parencite{wang2005residual}, which was proposed in an attempt to account for uncertainty in class assignment. However, the maximum probability rule is still believed to be able to minimise the number of incorrect assignments \parencite{goodman20071}. A Monte Carlo simulation study \parencite{bray2015eliminating} demonstrated that maximum probability assignment is less biased than multiple pseudo-class draws. 

Moreover, the Monte Carlo simulation study \parencite{bray2015eliminating} also found that an inclusive LCA (i.e. LCA with covariates) would probably perform better than non-inclusive LCA and has the potential to reduce bias of class assignment. However, whether this advantage can be extended to MLCA in the current project is unknown. Also, because the associations between the latent classes and distal outcomes (hypertension, and obesity) are still under exploration, appropriate covariates for an inclusive LCA model may not be known. Thus, as a first step, given the complexity of the NDNS RP dataset itself, we chose to fit the MLCA models without any covariate in either day level or individual level models. Future studies may be advised to consider incorporating other predictors in the MLCA models to see whether the classifications in both levels can be improved or not. 

Third, detailed information on the occupations of the participants was not available, and thus we could not account for those who had a shift or night work. But considering that completing a 4-consecutive-day food diary may be a tremendous burden for all participants, those with shift or night work might be less able to comply and complete their diary for 4 days. However, the latter hypothesis cannot be verified in the NDNS RP dataset. If people on shifts were included in the NDNS RP sample, the classification in both levels, as well as their associations with either hypertension or BMI would be biased. 

Furthermore, under-reporting due to the burden of recording food and drink intake for 4 days at each occasion is also a concern in the current survey. To evaluate the influence of under-reporting in the NDNS RP sample, the study team conducted a doubly labelled water (DLW) sub-study within the survey. The details of this sub-study can be accessed from the Appendix X of the Official Reports provided by Public Health England \parencite{bates2014national,roberts2018national,NDNSofficial}. In the sub-study, there is an assumption that in healthy adults, if energy consumed in food matches total energy expended over a given time period, the individuals are deemed to be in energy balance. Using the DLW technique, total energy expenditure (TEE) from a sub-sample of NDNS RP participants were measured and compared with the estimates of energy consumption from their 4 day diaries. Results of the sub-study showed reasonable agreement between energy consumed and TEE (overall ratio: 0.73), indicating some potential misreporting. Reasons such as misreporting of actual consumption, under-reporting or modified usual intake due to the burden of the survey/DLW sub-study may contribute to under-reporting of their energy consumed. We cannot extrapolate the estimated under-reporting to the whole sample since other individuals' diet might be differentially affected by under-reporting. Besides, within NDNS RP, it is not possible to differentiate between under-reporting due to ill health vs actual misreporting. Subsequently, most previous studies using data from the NDNS RP do not adjust for under-reporting of nutrients intake in the models.

Last, it should be noted that the findings from data-driven, exploratory methods may not be
generalizable to populations in other countries, since the carbohydrate eating patterns in both day level and individual level may only reflect the dietary habits of the current population. The latter reflect the socio-cultural and lifestyle characteristics of the adult population in the UK. Further research is warranted to explore and better understand the eating patterns in other populations. In particular, the cross-sectional study design prevented us from deducing any causal effect between the carbohydrate eating patterns and respective health outcomes including hypertension, BMI or abdominal obesity. 

Strengths of this study include the large, nationally representative sample of UK men and women. We applied a novel, objective approach using MLCA to examine the carbohydrate eating patterns while applying standardised criteria to determine the number of latent classes. The process of finding the classifications through model-based, data-driven procedure minimises reliance on researchers' preconceived notions of eating patterns. Eating patterns were determined from 4 consecutive days of well-designed, fully-examined food diaries that provided detailed information of what and when food and drinks were eaten. MLCA correctly accounted for the multilevel structure of the data, where the 4-day diet diaries were nested within the participants. Our findings also captured the day-to-day variation of the respective carbohydrate eating patterns within individuals. Moreover, the examination of associations between the individual level carbohydrate eating patterns and hypertension, BMI, abdominal obesity took the complex and the clustered survey design into consideration, in addition to accounting for participant non-response to survey. Health outcomes such as weight, height, waist circumferences, and blood pressure, diabetes status were measured objectively by trained nurses in addition to lipid profile, glucose, and HbA1C being measured in blood. These approaches have an advantage over self-reported measurements as they may minimise bias caused by misclassification and under-reporting.

\section{Conclusions}

We have successfully defined carbohydrate eating patterns in the general population in UK adults using the NDNS RP database in both observation day level and individual level. Low carbohydrate eaters tended to have higher energy intake from fat and alcohol compared to other types of carbohydrate eaters and eating later in the day. Moderate carbohydrate eaters reported the lowest total daily energy intake and tended to have a higher energy intake later in the day. High carbohydrate eaters obtained most of their carbohydrate as well as energy earlier in the day. These eating patterns specifically for carbohydrate intake were found to differ by timing, quantity, and contributions to energy consumption. Compared with low carbohydrate eaters, men with a moderate carbohydrate eating pattern may had a lower prevalence of hypertension. Women in moderate carbohydrate eating pattern and lived alone had a larger waist circumference compared with low carbohydrate eaters. Among women who lived with partners, a high carbohydrate eating pattern was associated with both lower BMI and smaller waist circumferences. Longitudinal studies are needed to investigate whether the identified eating patterns themselves are changing over time and to study how such circadian eating patterns may relate to the change of blood pressure, obesity, and other health outcomes (incidence of cancer, cardiovascular diseases, or mortality).