% Chapter Template

\chapter{Discussion and Conclusion} % Main chapter title

\label{Chapter 4} % Change X to a consecutive number; for referencing this chapter elsewhere, use \ref{ChapterX}





%----------------------------------------------------------------------------------------
%	SECTION 1
%----------------------------------------------------------------------------------------

\section{Main Findings}

\subsection{Carbohydrate eating patterns}

Using multilevel LCA as a novel technique, and the NDNS RP dietary database, this project examined carbohydrate eating temporal patterns firstly in the day level, based on which, individual level carbohydrate eating patterns were also defined subsequently. 

Among the dietary diaries collected, there were three distinct latent classes specifically for carbohydrate intake: 1) high probabilities of having high carbohydrate contained food across the hours of day (high percentage carbohydrate day); 2) low carbohydrate food dominant through out the hours of day (low percentage carbohydrate day); and 3) always having lunch and dinner day (regular meals day). And from these day level classifications and their features, one might anticipate that individuals who managed to follow the regular meals day might be eating a relatively healthier diet because it seemed to be a regular temporal eating habit; at this time point, we also believed that those who followed either high or low carbohydrate percentage days would probably consume higher total energy than those who followed most regular meals days. 

However, when the MLCA extended the model to individual level, three types of persons were further defined depending on their 4-day-diary: 1) low carbohydrate eaters, who mostly followed "regular meals day"; 2) moderate carbohydrate eaters, who had similar probabilities of following either "regular meals day" or "high percentage carbohydrate day"; 3) high carbohydrate eaters, who followed "high percentage carbohydrate day" for half of their survey. For the first time, as far as we know, the day-to-day food intake pattern variation within individuals was successfully captured by MLCA models. Results from the MLCA models showed that from the perspective of carbohydrate consumption, people were indeed changing their diet from day to day even within a short term period of survey. The MLCA models allowed the probability of following a certain type of carbohydrate eating day to vary across individuals. This properly accounted for the fact that for some people, their probability of following a type of food eating pattern during the survey could be higher/lower than that in the others. This finding also suggested that assuming a person will following a certain type of food intake pattern is not appropriate.

Surprisingly, low carbohydrate eaters whose dietary recordings suggested that they were mostly following a regular temporal meals pattern turned out to consume the highest amount of total energy among the three type of carbohydrate eaters. Detailed profiling of energy composition according to the time slots revealed that low carbohydrate eaters actually had higher proportion of energy contributed by both alcohol and fat. High percentage of fat consumption was shown in all 7 time slots, energy coming from alcohol exceeded more than one fifth of the total energy after 8 pm. These findings explained why they were actually consuming the highest energy among the three types of carbohydrate eaters. However, we also found that participants consuming low carbohydrate had higher prevalence of diabetes, hypertension, and obesity. These health issues might possibly lead them (or advices were given from their physicians) to replace carbohydrates in their diet to other energy sources such as fat, protein, or even alcohol. Therefore, there is a possibility that they chose to follow low carbohydrate diets out of healthy purpose, but many of them were replacing carbohydrates with higher energy condensed food or even alcohol at night which might indeed be a public health concern. 

Further more, looking into the details of the timing and composition of the energy intake among the moderate carbohydrate eaters. We realised that although moderate carbohydrate eaters did not consume as much alcohol as low carbohydrate eaters at night. But they were 




\subsection{Association between carbohydrate eating patterns and health outcomes}








%----------------------------------------------------------------------------------------
%	SECTION 2
%----------------------------------------------------------------------------------------




\section{Strengths and limitations}



\begin{itemize}
	\item MLCA ignored the order of observation days.
	\item We used the maximum probability rule and ignored that these are just probabilities.
\end{itemize}



\section{Conclusions}