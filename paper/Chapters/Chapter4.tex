% Chapter Template

\chapter{Discussion and Conclusion} % Main chapter title

\label{Chapter 4} % Change X to a consecutive number; for referencing this chapter elsewhere, use \ref{ChapterX}





%----------------------------------------------------------------------------------------
%	SECTION 1
%----------------------------------------------------------------------------------------

%\section{Main Findings}

\section{Carbohydrate eating patterns}

Using multilevel LCA as a novel technique, and the NDNS RP dietary database, this project examined carbohydrate eating temporal patterns firstly in the day level, based on which, individual level carbohydrate eating patterns were also defined subsequently. 

Among the dietary diaries collected, there were three distinct latent classes specifically for carbohydrate intake: 1) high probabilities of having high carbohydrate contained food across the hours of day (high percentage carbohydrate day); 2) low carbohydrate food dominant throughout the hours of day (low percentage carbohydrate day); and 3) always having lunch and dinner day (regular meals day). And from these day level classifications and their features, one might anticipate that individuals who managed to follow the regular meals day might be eating a relatively healthier diet because it seemed to be a regular temporal eating habit; at this time point, we also believed that those who followed either high or low carbohydrate percentage days would probably consume higher total energy than those who followed mostly regular meals days. 

However, when the MLCA extended the model to individual level, three types of persons were further defined depending on their 4-day-diary: 1) low carbohydrate eaters, who mostly followed "regular meals day"; 2) moderate carbohydrate eaters, who had similar probabilities of following either "regular meals day" or "high percentage carbohydrate day"; 3) high carbohydrate eaters, who followed "high percentage carbohydrate day" for half of their survey. For the first time, as far as we know, the day-to-day food intake pattern variation within individuals was successfully captured by MLCA models. Results from the MLCA models showed that from the perspective of carbohydrate consumption, people were indeed changing their diet from day to day even within a short-term period of survey. The MLCA models allowed the probability of following a certain type of carbohydrate eating day to vary across individuals. This properly accounted for the fact that for some people, their probability of following a type of food eating pattern during the survey could be higher/lower than that in the others. This finding also suggested that assuming a person will always follow a certain type of food intake pattern is not appropriate.

Surprisingly, low carbohydrate eaters whose dietary recordings suggested that they were mostly following a regular temporal meals pattern turned out to be consuming the highest amount of total energy among the three types of carbohydrate eaters. Detailed profiling of energy composition according to the time slots revealed that low carbohydrate eaters actually had a higher proportion of energy contributed by both alcohol and fat. A high percentage of fat consumption was shown in all 7-time slots, energy coming from alcohol exceeded more than one-fifth of the total energy after 8 pm. These findings explained why they were actually consuming the highest energy among the three types of carbohydrate eaters. In the meantime, we also found that participants consuming low carbohydrate food had a higher prevalence of diabetes, hypertension, and obesity. These health issues might possibly lead them (or advice were given by their physicians) to replace carbohydrates in their diet to other energy sources such as fat, protein, or even alcohol. Therefore, there is a possibility that they chose to follow low carbohydrate diets out of health purpose, but many of them were replacing carbohydrates with higher energy condensed food or even alcohol at night which might indeed be a public health concern. 

Next, when looking into the details of the timing and composition of the energy intake among the moderate carbohydrate eaters, we realised that although these individuals did not consume as much alcohol as low carbohydrate eaters did at night, they consumed the highest amount of energy, especially during the time period as late as after 10 pm. People fell into moderate carbohydrate eaters group seemed to have the tendency of having their food or meals later than high carbohydrate eaters. They consumed the highest amount of carbohydrates and also total energy among three types of carbohydrate eaters during the following time slots: 9-12 noon, 2-5 pm, 8-10 pm, and 10pm-6am. These individuals in the NDNS RP were younger, largely single, with lower average income, and lower education level. They might possibly correspond to the "late eaters" defined by previous studies \parencite{leech2017temporal, mansukhani2018investigating}. 

Finally, the high carbohydrate eaters identified by our MLCA models had the highest absolute total amount of carbohydrate intake. Most (nearly 70\%) of their total energy intake occurred during 6-9 am, 12-2 pm, and 5-8 pm time slots although their total energy consumption was not the highest (between low and moderate carbohydrate eaters). High carbohydrate eaters were also found to be the ones that consumed the least amount of energy after 8 pm. Therefore, contrary to what was anticipated at the beginning, people who followed high percentage carbohydrate days for most of their time were actually eating a healthier diet compared with the other two eating patterns. 

Our analyses looking for different temporal carbohydrate also highlighted the complexity of eating pattern behaviours in the population and the utility of exploratory, data-driven methods to objectively identify eating patterns that reflect both timing and quantities of food intake, which may not have been detected so far in the literature.


\section{Associations between carbohydrate eating patterns and health outcomes}

Among men, who were classified as moderate carbohydrate eaters probably had lower odds of having hypertension, after adjustment of age, live with a partner or not, educational level, BMI, smoking status and total energy intake. As discussed above, moderate carbohydrate eaters turned to have meals (or energy consumption) later in time compared with high carbohydrate eaters. But, it is noteworthy that low carbohydrate eaters also consumed a large amount of energy (from both fat and alcohol) at night. Therefore, considering that moderate carbohydrate eaters were younger than low carbohydrate eaters (although age was adjusted in the full models), there is probably reverse causality exists here (also due to the nature of cross-sectional study). That is, they were potentially both late eaters, however, with increased age (and so as increased health-related problems/concern) some of them modified their habits, such as quitting smoking, replacing carbohydrate food with other energy sources (so that they became low carbohydrate eaters with hypertension) which lead to the phenomenon of lower odds of hypertension in moderate carbohydrate eaters. Although these hypotheses cannot be determined by the cross-sectional data from NDNS RP, if they were true, the energy sources they used to replace carbohydrate in their diet were apparently not very wisely chosen. 

Among women, whether living with a partner became an interaction factor for the associations between carbohydrate eating patterns and BMI and abdominal obesity (WC). Directions of the associations were opposite to each other depending on whether women were living with a partner or not. This interaction effect was more obvious when looking at abdominal obesity measurement. High carbohydrate eaters had lower BMI and WC in those lived with their partners, while moderate carbohydrate eaters had higher WC in those who lived alone after adjustment of age, education level, smoking status, total energy intake, and alcohol consumption. High carbohydrate eaters who were characterised with high and early in the daily energy consumption and low fat and alcohol intake may reflect a healthier diet and lifestyle, but this might be different between women who lived alone and those who lived with their partners. It was often assumed that live alone may associate with a lower diversity of food intake, and a higher likelihood of having an unhealthy dietary choice \parencite{hanna2015relationship}. Therefore, there may be differences in the actual contents consumed in their high carbohydrates, or there may be other social, psychological or lifestyle-related factors related with living alone which were not measured or did not include in the models. Thus, the inverse association between high carbohydrate eating pattern and BMI or abdominal obesity were only observed among women who lived with partners, further investigation of this hypothesis is needed. Whereas the reason why moderate carbohydrate eaters' WC was larger than low carbohydrate eaters only among women who lived alone is unknown, given that the evidence of this association was weak and borderline significant, whether it was just a false positive finding should also be explored in other studies.

%but if the reverse causality reason still applies, women who lived alone may be less likely to modify their diet (replacing carbohydrate with fat or alcohol) and therefore they stayed 




%Women lived with partners had higher proportion of being classified as high carbohydrate eaters 



%----------------------------------------------------------------------------------------
%	SECTION 2
%----------------------------------------------------------------------------------------




\section{Limitations and strengths}

There are several limitations in the current project that merit consideration.  

Firstly, we ignored the order of observation days in the MLCA models. The food consumption diaries were accomplished by participants for at least 3 out of 4 consecutive days. In the multilevel analyses, these 3 or 4 days' observations were treated as if they were exchangeable in the models. Since one's diet might change according to the season, day of the week (weekdays or weekends), or sometimes depends on what they had consumed the day before, exchangeability of their diaries is a strong assumption and cannot be overcome by the MLCA models adopted here. Other statistical techniques, which could take the order or the longitudinal nature of the data into account, such as repeated measures latent class analysis (RMLCA), latent transition analysis (LTA) \parencite{collins2010latent}, or latent class growth analysis (LCGA) \parencite{davidian2008growth,jung2008introduction,andruff2009latent} are not applicable for the NDNS RP dataset. For the purpose of maintaining the response rate, flexibility was allowed for the NDNS RP participants in their choice of which day to begin their diary. But the above mentioned alternative models require that the 3 or 4 repeated measurements of food diaries be recorded at the same time points longitudinally. Specifically, LCGA models (also called growth mixture model) will also need to model the change of the odds of the probability over time as a function (quadratic, or cubic) which is apparently not the objective in the current project. 

Secondly, the classification of individuals to latent carbohydrate eating classes was defined by maximum posterior probability assignment rule. This approach assigns individuals to the class for which they have the highest posterior probability of membership \parencite{nagin2005group}. The other approach is multiple pseudo-class draws \parencite{wang2005residual}, which was proposed in an attempt to account for uncertainty in class assignment. However, the maximum probability rule is still believed to be able to minimize the number of incorrect assignments \parencite{goodman20071}. A Monte Carlo simulation study \parencite{bray2015eliminating} demonstrated that maximum probability assignment is less biased than multiple pseudo-class draws. Moreover, this simulation study also found that an inclusive LCA (i.e. LCA with covariates) would probably perform better than non-inclusive LCA, and has the potential to reduce bias of class assignment. However, whether this advantage can be extended to MLCA in the current project is unknown, and also due to the reason that the associations between the latent classes and distal outcomes (hypertension, and BMI) in our analyses are mostly still under exploration, therefore, the model for LCA with covariates can never be well-known before our analyses. Thus, as a first step, given the complexity of the NDNS RP dataset itself, we chose to fit the MLCA models without any covariate in either day level or individual level models. Future studies may be advised to consider incorporating other predictors in the MLCA models to see whether the classifications in both levels can be improved or not. 

Thirdly, information regarding the detailed occupations of the participants was not available and because of which we could not exclude those who had a shift work involving night work. But considering that completing a 4-consecutive-day food diary would be a large burden for the participants, those had a shift work involving night work might possibly perform with relatively lower compliance and could not complete their diary for at least 3 days. But this hypothesis cannot be verified in the NDNS RP dataset. If people working with shift work were included in the NDNS RP sample, the classification in both levels, as well as their associations with either hypertension or BMI would be biased. Furthermore, under-reporting due to the burden of writing down everything they eat at each occasion is also a concern in the food diaries collected. To evaluate the influence of under-reporting in the NDNS RP sample, the study team of NDNS RP conducted doubly labelled water (DLW) sub-study within the survey. The details of this sub-study can be accessed from the Appendix X of the Official Reports provided by Public Health England \parencite{bates2014national,roberts2018national,NDNSofficial}. Briefly, in healthy adults, if, for a given period of time, energy consumed in food matches total energy expended, they are in energy balance. In NDNS RP sub-study, estimates of energy intake (EI) from the four-day diary were compared with measurements of total energy expenditure (TEE) using the DLW technique in a sub-sample of survey participants. Results of the sub-study showed reasonable agreement between EI and TEE (overall ratio: 0.73). Reasons such as misreporting of actual consumption; under-reported or modified usual intake due to the burden of the survey/DLW sub-study are potential factors may contribute to under-reporting of their EI. We cannot extrapolate the estimated under-reporting to the whole sample since other individuals' diet might be affected by under-reporting differentially. 

Lastly, it should be noted that findings from data-driven, exploratory methods may not be
generalizable to populations in other countries. Since the carbohydrate eating patterns in both day level and individual level probably only represent the socio-cultural and lifestyle characteristics of the adult population in the UK. Further researches are warranted to explore and better understand the eating patterns in other populations. Besides, the cross-sectional study design also refrained us from deducing any causal effect between the carbohydrate eating patterns and hypertension, BMI or abdominal obesity. 

Strengths of this study include the large, nationally representative sample of men and women in the UK. We applied a novel, objective approach using MLCA to examine the carbohydrate eating patterns while applying standardised criteria to determine the numbers of latent classes. The process of finding the classifications through model-based, data-driven procedure minimises reliance on researchers' preconceived notions of eating patterns. Eating patterns were determined from 4 consecutive days of well-designed, fully-examined food diaries. MLCA correctly accounted for the multilevel structure of the data which means that the 4-day diet diaries were nested within the participants. Our findings also captured the day-to-day variation of following different carbohydrate eating patterns within individuals. Moreover, the examination of associations between the individual level carbohydrate eating patterns and hypertension, BMI, abdominal obesity were conducted correctly considering the probability of participant selection and the clustered survey design. Health outcomes such as weight, height, waist circumferences, and blood pressure, diabetes status were measured by trained nurses and blood tests which could also minimise bias caused by misclassification and under-reporting of these measurements.

\section{Conclusions}

We have successfully defined carbohydrate eating patterns in the general population in the UK adults using the NDNS RP database in both observation day level and individual level. Low carbohydrate eaters turned out to have more energy that contributed by both fat and alcohol. Moderate carbohydrate eaters consumed the lowest total energy, while they had the tendency of having meals later in time-of-day. High carbohydrate eaters consumed most of their carbohydrate as well as energy earlier in time-of-day. These dietary patterns specifically for carbohydrate intake were found to be differed by timing, quantity, and contributions to energy consumption. Compared with low carbohydrate eaters, men had moderate carbohydrate eating pattern may associate with a lower prevalence of hypertension, women in this latent class who lived alone may associate with a larger waist circumference. Among women who lived with partners, high carbohydrate eating pattern was associated with both lower BMI and smaller waist circumferences. Longitudinally designed studies are needed in finding whether the eating patterns themselves are changing over time and in investigating how such circadian eating patterns may relate to the change of blood pressure, obesity, and other health outcomes (incidence of cancer, cardiovascular diseases, or perhaps mortality).