% Chapter Template

\chapter{Discussion and Conclusion} % Main chapter title

\label{Chapter 4} % Change X to a consecutive number; for referencing this chapter elsewhere, use \ref{ChapterX}





%----------------------------------------------------------------------------------------
%	SECTION 1
%----------------------------------------------------------------------------------------

%\section{Main Findings}

\section{Carbohydrate eating patterns}

Using multilevel LCA as a novel technique, and the NDNS RP dietary database, this project examined carbohydrate eating temporal patterns firstly in the day level, based on which, individual level carbohydrate eating patterns were also defined subsequently. 

Among the dietary diaries collected, there were three distinct latent classes specifically for carbohydrate intake: 1) high probabilities of having high carbohydrate contained food across the hours of day (high percentage carbohydrate day); 2) low carbohydrate food dominant through out the hours of day (low percentage carbohydrate day); and 3) always having lunch and dinner day (regular meals day). And from these day level classifications and their features, one might anticipate that individuals who managed to follow the regular meals day might be eating a relatively healthier diet because it seemed to be a regular temporal eating habit; at this time point, we also believed that those who followed either high or low carbohydrate percentage days would probably consume higher total energy than those who followed mostly regular meals days. 

However, when the MLCA extended the model to individual level, three types of persons were further defined depending on their 4-day-diary: 1) low carbohydrate eaters, who mostly followed "regular meals day"; 2) moderate carbohydrate eaters, who had similar probabilities of following either "regular meals day" or "high percentage carbohydrate day"; 3) high carbohydrate eaters, who followed "high percentage carbohydrate day" for half of their survey. For the first time, as far as we know, the day-to-day food intake pattern variation within individuals was successfully captured by MLCA models. Results from the MLCA models showed that from the perspective of carbohydrate consumption, people were indeed changing their diet from day to day even within a short term period of survey. The MLCA models allowed the probability of following a certain type of carbohydrate eating day to vary across individuals. This properly accounted for the fact that for some people, their probability of following a type of food eating pattern during the survey could be higher/lower than that in the others. This finding also suggested that assuming a person will always follow a certain type of food intake pattern is not appropriate.

Surprisingly, low carbohydrate eaters whose dietary recordings suggested that they were mostly following a regular temporal meals pattern turned out to consume the highest amount of total energy among the three types of carbohydrate eaters. Detailed profiling of energy composition according to the time slots revealed that low carbohydrate eaters actually had higher proportion of energy contributed by both alcohol and fat. High percentage of fat consumption was shown in all 7 time slots, energy coming from alcohol exceeded more than one fifth of the total energy after 8 pm. These findings explained why they were actually consuming the highest energy among the three types of carbohydrate eaters. However, we also found that participants consuming low carbohydrate food had higher prevalence of diabetes, hypertension, and obesity. These health issues might possibly lead them (or advices were given from their physicians) to replace carbohydrates in their diet to other energy sources such as fat, protein, or even alcohol. Therefore, there is a possibility that they chose to follow low carbohydrate diets out of health purpose, but many of them were replacing carbohydrates with higher energy condensed food or even alcohol at night which might indeed be a public health concern. 

Next, when looking into the details of the timing and composition of the energy intake among the moderate carbohydrate eaters, we realised that although these individuals did not consume as much alcohol as low carbohydrate eaters at night, they consumed the highest amount of energy, especially during time period as late as after 10 pm. People fell into moderate carbohydrate eaters group seemed to have the tendency of having their food or meals later than the other two types of carbohydrate eaters. They consumed the highest amount of carbohydrates and also total energy among three types of carbohydrate eaters during the following time slots: 9-12 noon, 2-5 pm, 8-10 pm, and 10pm-6am. These individuals in the NDNS RP were younger, mostly single, with lower average income, and lower education level. They might possibly correspond to the "late eaters" defined by previous studies \parencite{leech2017temporal, mansukhani2018investigating}. 

Lastly, the high carbohydrate eaters identified by our MLCA models had the highest absolute total amount of carbohydrate intake. Most of their energy intake occured during 6-9 am, 12-2 pm, and 5-8 pm time slots while their average energy consumption was not the highest (between low and moderate carbohydrate eaters). High carbohydrate eaters were also found to be the ones that consumed the least amount of energy after 8 pm. Therefore, contrary to what was anticipated at the beginning, people who followed high percentage carbohydrate days for most of their time, were actually eating a healthier diet compared with the other two eating patterns. 

Our analyses looking for different temporal carbohydrate also highlighted the complexity of eating pattern behaviours in the population and the utility of exploratory, data-driven methods to objectively identify eating patterns that reflect both timing and quantities of food intake, which may have not been detected so far in the literature.


\section{Association between carbohydrate eating patterns and health outcomes}

Among men, who were classified as moderate carbohydrate eaters probably had lower odds of having hypertension, after adjustment of age, live with partner or not, educational level, BMI, smoking status and total energy intake. As discussed above, moderate carbohydrate eaters turned to have meals (or energy consumption) later in time compared with high carbohydrate eaters, but low carbohydrate eaters also consumed large amount of energy (from both fat and alcohol) at night. Therefore, considering that moderate carbohydrate eaters were younger than low carbohydrate eaters (although age was adjusted in the full models), there is probably reverse causality exists here (also due to the nature of cross-sectional study). That is, they were potentially both late eaters, however, with their age increased (and so as increased health-related problems/concern) some of them modified their habits, such as quit smoking, replace carbohydrate food with other energy sources which lead to the phenomenon of lower odds of hypertension in moderate carbohydrate eaters. Although the hypothesis cannot be determined by NDNS RP's cross-sectional data, if these theories were true, the energy sources they used to replace carbohydrate in their diet were apparently not very wisely chosen. 

Among women, whether living with a partner became an interaction factor for the associations between carbohydrate eating patterns and BMI and abdominal obesity (WC). Directions of the associations were opposite to each other depending on whether women were living with a partner or not. This interaction effect was more obvious when looking at abdominal obesity measurement. High carbohydrate eaters had lower BMI and WC in those lived with their partner, while moderate carbohydrate eaters had higher WC in those who lived alone after adjustment of age, education level, smoking status, total energy intake, and alcohol consumption. High carbohydrate eaters who were characterised with high and early in the day energy consumption and low fat and alcohol intake may reflect a healthier diet and lifestyle, but this might be different between women who lived alone and those who lived with their partners. It was often assumed that live alone may associated with lower diversity of food intake, and a higher likelihood of having an unhealthy dietary pattern \parencite{hanna2015relationship}. Therefore, there may be differences in the actual contents consumed in the high carbohydrates eaters, or there may be other social, psychological or lifestyle related factors related with living alone which we did not included in the models, so that the inverse association between high carbohydrate eating pattern and BMI or abdominal obesity were only observed among women who lived with partners, further analysis is needed. Whereas the reason why moderate carbohydrate eaters' WC was larger than low carbohydrate eaters only among women who lived alone is unknown, given that the evidence of this association was weak and borderline significant, whether it was just a false positive result should be explored in other studies.

%but if the reverse causality reason still applies, women who lived alone may be less likely to modify their diet (replacing carbohydrate with fat or alcohol) and therefore they stayed 




%Women lived with partners had higher proportion of being classified as high carbohydrate eaters 



%----------------------------------------------------------------------------------------
%	SECTION 2
%----------------------------------------------------------------------------------------




\section{Strengths and limitations}



\begin{itemize}
	\item MLCA ignored the order of observation days.
	\item We used the maximum probability rule and ignored that these are just probabilities. but inclusive MLCA might perform better.
	\item we do not have the information of whether the participants were doing job that requires shift work.
	\item dietary report under estimation
	\item findings may not be generalizable to populations from other countries.
\end{itemize}



\section{Conclusions}

We have successfully defined carbohydrate eating patterns in the general population in the UK adults using the NDNS RP database in both observation day level and participant level. Low carbohydrate eaters turned to have more energy that contributed by both fat and alcohol. Moderate carbohydrate eaters consumed the lowest total energy, while they had the tendency of having meals later in time-of-day. High carbohydrate eaters consumed most of their carbohydrate as well as energy earlier in time-of-day. These dietary patterns specifically for carbohydrate intake were found to be differed by timing, quantity and resources of energy consumption. Compared with low carbohydrate eaters, men had moderate carbohydrate eating pattern may associated with a lower prevalence of hypertension, women in this latent class who lived alone may associated with a larger waist circumferences. Among women who lived with partners, high carbohydrate eating pattern was associated with both lower BMI and smaller waist circumferences. Future studies exploring how such carbohydrate eating patterns may relate longitudinally to change of obesity, hypertension and diabetes incidence and other health outcomes are needed.