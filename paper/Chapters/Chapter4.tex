% Chapter Template

\chapter{Discussion and Conclusion} % Main chapter title

\label{Chapter 4} % Change X to a consecutive number; for referencing this chapter elsewhere, use \ref{ChapterX}





%----------------------------------------------------------------------------------------
%	SECTION 1
%----------------------------------------------------------------------------------------

\section{Main Findings}

\subsection{Carbohydrate eating patterns}

Using multilevel LCA as a novel technique, and the NDNS RP dietary database, this project examined carbohydrate eating temporal patterns in the day level, based on which, individual level carbohydrate eating patterns were also defined subsequently. Among all of the dietary diaries, there were three distinct latent classes specifically for carbohydrate intake: 1) high probabilities of having high carbohydrate contained for across the hours of day (high carbohydrate day); 2) low carbohydrate food dominant through out the hours of day (low carbohydrate day); and 3) always having lunch and dinner day (regular meals day). 

For the first time, as far as we know, the non-parametric MLCA models successfully captured the day-to-day variability within individuals as expected and showed that from the perspective of carbohydrate eating pattern people were switching between different types of carbohydrate consumptions. The MLCA models allowed the probability of a certain type of carbohydrate eating day to vary across individuals. This properly accounted for the fact that for some people, their probability of following a regular meals day during the survey was much higher than the others (low carbohydrate eaters); and for some people, their probability of following high carbohydrate eating day during their survey were quite high (high carbohydrate eaters); while for the rest of the participants, they had similar probabilities of following either regular meals day or high carbohydrate eating day (moderate carbohydrate eaters). 




\subsection{Association between carbohydrate eating patterns and health outcomes}








%----------------------------------------------------------------------------------------
%	SECTION 2
%----------------------------------------------------------------------------------------




\section{Strengths and limitations}



\begin{itemize}
	\item MLCA ignored the order of observation days.
	\item We used the maximum probability rule and ignored that these are just probabilities.
\end{itemize}



\section{Conclusions}