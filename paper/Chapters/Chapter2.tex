% Chapter Template

\chapter{Methods} % Main chapter title

\label{Chapter 2} % Change X to a consecutive number; for referencing this chapter elsewhere, use \ref{ChapterX}

%----------------------------------------------------------------------------------------
%	SECTION 1
%----------------------------------------------------------------------------------------

\section{Strategy}


%-----------------------------------
%	SUBSECTION 1
%-----------------------------------
\subsection{Latent class analysis (LCA)}


%-----------------------------------
%	SUBSECTION 2
%-----------------------------------

\subsection{Multilevel latent class analysis (MLCA)}\vspace{-0.3cm}


Multilevel latent class analysis accounts for the nested structure of the data by allowing latent class intercepts to vary across level 2 units and thereby examining if and how level 2 units influence the level 1 latent classes. These random intercepts allow the probability of
membership in a particular level 1 (observations) latent class to vary across level 2 units (e.g., here in the current context are the individuals). Essentially this allows the probability that an observation will belong to a
particular level 1 latent class to vary across Level 2 units (individuals). \vspace{-0.5cm}

\subsubsection{Parametric approach}\vspace{-0.3cm}

Proposed by Vermunt \parencite{Vermunt, vermunt2008latent} and Asparouhov and Muth\'en \parencite{muthen2009multilevel},  a traditional, parametric approach can be applied using a logistic regression model. In an unconditional logistic regression model, the probability of the outcome (i.e. being in latent class $k$) is constant within the 4-day survey for each individual (level 2). Therefore, say when we are fitting a model with $k (k = 1, \cdots, K)$ latent classes in level 1, then in each individual (level 2), there is a probability of being in latent class $k$. A random effect model consider the individual (level 2) to be drawn from a population of adults in the UK, and the probability of the outcome (i.e. being in latent class $k$) across individuals is considered to be a random variable \parencite{snijders2011multilevel}. The 2-level random intercept effect regression model can be expressed as:\vspace{-0.4cm}


\begin{equation}
\begin{aligned}
\text{logit}[P(C_{ij} = t)] & = \beta_{0j} & \textbf{(level 1)}  \\
\beta_{0j} & = \gamma_0 + \gamma_1 w_j + u_{0j} & \textbf{(level 2)} \\ 
\Rightarrow P(C_{ij} = t) & = \frac{\exp{(\gamma_0 + \gamma_1 w_j + u_{0j})}}{1 + \exp{(\gamma_0 + \gamma_1 w_j + u_{0j})}} \\
\end{aligned}
\label{randomLCA}
\end{equation}
\vspace{-0.3cm}

Where we define: 

\begin{itemize}
	\item $P(C_{ij} = t)$ as the probability that the observation is belonging to latent class $t$;
	\item $u_{0j}$ as the random intercept for $j$th individual; 
	\item the random intercept are assumed be normally distributed (i.e. $u_{0j} \sim N(0, \sigma_{u_0}^2)$), the magnitude of the $u_{0j}$ variance ($\sigma_{u_0}^2$) indicates the influence of the individuals (level 2);
	\item $w_j$ is the predictor for individual (level 2), such as age, and/or sex.
\end{itemize}


Same as in the typical LCA models, the latent class variable in a MLCA is defined by multiple observed indicators (here is defined by the responses of eating carbohydrates within each hour, over 24 hours and during 4 consecutive days of survey period). Considering that the latent class indicators are indicator variables ($U_{ijk}$), the MLCA model can be written as follows:\vspace{-0.8cm}

\begin{equation}
%\begin{aligned}
P(U_{ij1} = s_1, U_{ij2} = s_2, \cdots, U_{ijk} = s_{k}) = \sum_{t=1}^{T}P(C_{ij}=t)\prod_{k=1}^{K}P(U_{ijk} = s_k | C_{ij} = t)
%\end{aligned}
\label{MLCA}
\end{equation}
\vspace{-0.8cm}


Where, 

\begin{itemize}
	\item $ U_{ijk} $ represents the response of eating \textcolor{red}{high/low} carbohydrates on $i$th day of the survey ($i \in (1,2,3,4)$) in subject $j$ (level 2) at the $k$th hour of the day ($k \in (1, 2, 3, \cdots, 24)$);
	\item $C_{ij}$ denotes the latent class membership for subject $j$ on $i$th day of the survey;
	\item A specific latent class is referred to as $t$, and the total number of level 1 latent classes is denoted by $T$;
	\item $P(U_{ijk} = s_k|C_{ij} = t)$ is the probability of a specific response pattern, conditional on membership in latent class $t$.
\end{itemize}


The $P(C_{ij} = t)$ in equation \ref{MLCA} is what we have already defined in equation \ref{randomLCA}.

\vspace{-0.5cm}
\subsubsection{Non-Parametric approach}\vspace{-0.3cm}

Another approach is a non-parametric MLCA. The $T-1$

%----------------------------------------------------------------------------------------
%	SECTION 2
%----------------------------------------------------------------------------------------

\section{Survey Data}

%-----------------------------------
%	SUBSECTION 1
%-----------------------------------
\subsection{Survey Selection Method}


%-----------------------------------
%	SUBSECTION 2
%-----------------------------------
\subsection{Response rates}
